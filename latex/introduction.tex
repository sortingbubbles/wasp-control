\section{Εισαγωγή}
Η νοημοσύνη μπορεί να οριστεί ως η ικανότητα προσαρμογής της συμπεριφοράς ενός συστήματος σε ένα περιβάλλον που συνεχώς αλλάζει. Σύμφωνα με τον πατέρα της Τεχνητής Νοημοσύνης, Alan Turing, η μορφή και η εξωτερική εμφάνιση ενός συστήματος δεν έχει καμία σχέση με τη νοημοσύνη του.

Είναι γνωστό όμως ότι οι άνθρωποι παρουσιάζουν στοιχεία ευφυούς συμπεριφοράς. Οι άνθρωποι, είναι επίσης προϊόντα της φυσικής εξέλιξης και έτσι, με την μοντελοποίηση της εξελικτικής διαδικασίας, θα μπορούσε κανείς να φιλοδοξεί ότι θα δημιουργήσει μια ευφυή συμπεριφορά.

Οι εξελικτικοί αλγόριθμοι προσομοιώνουν την εξέλιξη στον υπολογιστή. Το αποτέλεσμα αυτής της προσομοίωσης είναι μια σειρά από αλγόριθμους βελτιστοποίησης, οι οποίοι βασίζονται συνήθως σε μια συλλογή από απλούς κανόνες. Η επαναληπτική βελτιστοποίηση βελτιώνει την ποιότητα των λύσεων μέχρι να βρεθεί η πιο βέλτιστη ή τουλάχιστον εφικτή λύση.

Ίσως να μην είναι ξεκάθαρη η νοημοσύνη της εξέλιξης, λόγο του ότι στη φύση χρειάζεται να περάσει αρκετός καιρός για να παρατηρηθούν οι σημαντικές αλλαγές σε κάποιο σύστημα. Ένας οργανισμός αναπτύσσει μια συμπεριφορά για κάποια στοιχεία του περιβάλλοντος που δεν έχει ξανασυναντήσει. Σε περίπτωση που επιζήσει μετά τις αλλαγές του περιβάλλοντος αυτή η συμπεριφορά κληρονομείται από τους απογόνους του και έτσι βγαίνει το συμπέρασμα ότι ένας οργανισμός έχει τη δυνατότητα της μάθησης καις της πρόβλεψης των αλλαγών στο περιβάλλον. Ολόκληρη η παραπάνω διαδικασία ίσως να φανεί πολύ αργή για τον άνθρωπο, αυτό όμως δεν ισχύει για τους υπολογιστές, καθώς η προσομοίωση της εξέλιξης δεν απαιτεί εκατομμύρια χρόνια.

Η εξελικτική προσέγγιση της μηχανικής μάθησης βασίζεται σε υπολογιστικά μοντέλα της φυσικής επιλογής και της γενετικής. Αυτή η προσέγγιση ονομάζεται εξελικτικοί αλγόριθμοι και συνδυάζει τεχνικές όπως οι γενετικοί αλγόριθμοι, οι εξελικτικές στρατηγικές και ο γενετικός προγραμματισμός. Όλες αυτές οι τεχνικές προσομοιώνουν την εξέλιξη με τη χρήση των διαδικασιών της επιλογής, της μετάλλαξης και του ανασυνδυασμού.

 