%%%%%%%%%%%%%%%%%%%%%%%%%%%%%%%%%%%%%%%%%
% Beamer Presentation
% LaTeX Template
% Version 1.0 (10/11/12)
%
% This template has been downloaded from:
% http://www.LaTeXTemplates.com
%
% License:
% CC BY-NC-SA 3.0 (http://creativecommons.org/licenses/by-nc-sa/3.0/)
%
%%%%%%%%%%%%%%%%%%%%%%%%%%%%%%%%%%%%%%%%%

%----------------------------------------------------------------------------------------
%	PACKAGES AND THEMES
%----------------------------------------------------------------------------------------

\documentclass[xetex,mathserif,serif,14pt]{beamer}

\mode<presentation> {

% The Beamer class comes with a number of default slide themes
% which change the colors and layouts of slides. Below this is a list
% of all the themes, uncomment each in turn to see what they look like.

%\usetheme{default}
%\usetheme{AnnArbor}
%\usetheme{Antibes}
%\usetheme{Bergen}
%\usetheme{Berkeley}
%\usetheme{Berlin}
%\usetheme{Boadilla}
%\usetheme{CambridgeUS}
%\usetheme{Copenhagen}
%\usetheme{Darmstadt}
%\usetheme{Dresden}
%\usetheme{Frankfurt}
%\usetheme{Goettingen}
\usetheme[hideothersubsections]{Hannover}
%\usetheme{Ilmenau}
%\usetheme{JuanLesPins}
%\usetheme{Luebeck}
%\usetheme{Madrid}
%\usetheme{Malmoe}
%\usetheme{Marburg}
%\usetheme{Montpellier}
%\usetheme{PaloAlto}
%\usetheme{Pittsburgh}
%\usetheme{Rochester}
%\usetheme{Singapore}
%\usetheme{Szeged}
%\usetheme{Warsaw}

% As well as themes, the Beamer class has a number of color themes
% for any slide theme. Uncomment each of these in turn to see how it
% changes the colors of your current slide theme.

%\usecolortheme{albatross}
%\usecolortheme{beaver}
%\usecolortheme{beetle}
%\usecolortheme{crane}
%\usecolortheme{dolphin}
%\usecolortheme{dove}
%\usecolortheme{fly}
%\usecolortheme{lily}
%\usecolortheme{orchid}
%\usecolortheme{rose}
%\usecolortheme{seagull}
%\usecolortheme{seahorse}
%\usecolortheme{whale}
%\usecolortheme{wolverine}
\usecolortheme[named=violet]{structure}

%\setbeamertemplate{footline} % To remove the footer line in all slides uncomment this line
%\setbeamertemplate{footline}[page number] % To replace the footer line in all slides with a simple slide count uncomment this line

%\setbeamercolor{frametitle}{fg=red,bg=red!20} % Redefine color of frame title box
%\setbeamercolor*{title}{bg=red,fg=white} % Redefine color of presentation title box
%\setbeamercolor{block title}{fg=black,bg=black!20} % Redefine color of block title %bg=background, fg=foreground
%\setbeamercolor{block body}{fg=black,bg=red!15} % Redefine color of block body

\makeatletter
\setbeamertemplate{footline} % Redefine footer (mainly width of boxes)
{
  \leavevmode%
  \hbox{%
  \begin{beamercolorbox}[wd=.41\paperwidth,ht=2.25ex,dp=1ex,center]{author in head/foot}%
    \usebeamerfont{author in head/foot}\insertshortauthor~~(\insertshortinstitute)
  \end{beamercolorbox}%
  \begin{beamercolorbox}[wd=.34\paperwidth,ht=2.25ex,dp=1ex,center]{title in head/foot}%
    \usebeamerfont{title in head/foot}\insertshorttitle
  \end{beamercolorbox}%
  \begin{beamercolorbox}[wd=.25\paperwidth,ht=2.25ex,dp=1ex,right]{date in head/foot}%
    \usebeamerfont{date in head/foot}\insertshortdate{}\hspace*{2em}
    \insertframenumber{} / \inserttotalframenumber\hspace*{2ex}
  \end{beamercolorbox}}%
  \vskip0pt%
}
\makeatother

\setbeamertemplate{navigation symbols}{} % To remove the navigation symbols from the bottom of all slides uncomment this line
}

\usepackage{polyglossia} % χρησιμοποιείται για καλύτερη υποστήριξη των Ελληνικών
\usepackage{graphicx} % Allows including images
\usepackage{booktabs} % Allows the use of \toprule, \midrule and \bottomrule in tables
\usepackage{pgfplots} % For drawing the queries
\usepgfplotslibrary{groupplots}
\usetikzlibrary{pgfplots.groupplots}
\usepackage{xcolor} % For colors

\pgfplotsset{compat=1.10}

\setmainlanguage[numerals=arabic]{greek} % κύρια γλώσσα
\setotherlanguages{english} % δευτερεύουσα γλώσσα

%----------------------------------------------------------------------------------------
%	TITLE PAGE
%----------------------------------------------------------------------------------------

\title[Εξόντωση Σφηκών]{Εξόντωση Σφηκών με Χρήση Εξελικτικών Αλγορίθμων} % The short title appears at the bottom of every slide, the full title is only on the title page

\author[Χατσατριάν, Κοσματόπουλος]{Άνι Χατσατριάν, Μιχάλης Κοσματόπουλος} % Your name
\institute[ΑΤΕΙΘ] % Your institution as it will appear on the bottom of every slide, may be shorthand to save space
{
Αλεξάνδρειο Τεχνολογικό Εκπαιδευτικό Ίδρυμα Θεσσαλονίκης \\ % Your institution for the title page
\medskip
\textit{\{achatsat, mkosm\}@it.teithe.gr} % Your email address
}
\date{23 Μαΐου 2014} % Date, can be changed to a custom date

\newfontfamily\greekfont[Script=Greek]{Linux Libertine O} % work-around για bug του polyglossia
\setmainfont[Kerning=On,Mapping=tex-text]{Linux Libertine O} % roman font

\begin{document}

\begin{frame}
\titlepage % Print the title page as the first slide
\end{frame}

%\begin{frame}
%\frametitle{Overview} % Table of contents slide, comment this block out to remove it
%\tableofcontents % Throughout your presentation, if you choose to use \section{} and \subsection{} commands, these will automatically be printed on this slide as an overview of your presentation
%\end{frame}

%----------------------------------------------------------------------------------------
%	PRESENTATION SLIDES
%----------------------------------------------------------------------------------------

%------------------------------------------------
\section{Εισαγωγή} % Sections can be created in order to organize your presentation into discrete blocks, all sections and subsections are automatically printed in the table of contents as an overview of the talk
%------------------------------------------------

\begin{frame}
\frametitle{Εισαγωγή}
Στόχος της εργασίας είναι η εύρεση της βέλτιστης λύσης ενός προβλήματος με τη χρήση ενός εξελικτικού αλγορίθμου.

\end{frame}

\section{Βιολογικό Υπόβαθρο}

\subsection{Η Θεωρία της Εξέλιξης}

\begin{frame}
\frametitle{Η Θεωρία της Εξέλιξης}
\begin{itemize}
  \item Η ικανότητα των οργανισμών να προσαρμόζονται στις αλλαγές του περιβάλλοντος και να αποκτούν απογόνους.
  \item Οι αποκλίσεις στους φαινότυπους των οργανισμών.
\end{itemize}
\end{frame}

\subsection{Βιολογικά Χαρακτηριστικά}

\begin{frame}
\frametitle{Βιολογικά Χαρακτηριστικά}
    \begin{itemize}
      \item Γονίδια
      \item Alleles
      \item Χρωμόσωμα
      \item Γονότυπος
      \item Φαινότυπος      
    \end{itemize}
\end{frame}

\section{Μοντέλα Εξελικτικών Αλγορίθμων}

\subsection{Γενική μορφή ΕΑ}

\begin{frame}
\frametitle{Γενική μορφή ΕΑ}
Τι είναι;
\end{frame}

\subsection{Γενετικοί Αλγόριθμοι}

\begin{frame}
\frametitle{Γενετικοί Αλγόριθμοι}
Τι είναι;
\end{frame}

\subsection{Τρόποι Αναπαράστασης}

\begin{frame}
\frametitle{Τρόποι Αναπαράστασης}
Τι είναι;
\end{frame}

\subsection{Δημιουργία και Αξιολόγηση Πληθυσμού}

\begin{frame}
\frametitle{Δημιουργία και Αξιολόγηση Πληθυσμού}
Τι είναι;
\end{frame}

\subsection{Επιλογή Γονέων}

\begin{frame}
\frametitle{Γενετικοί Αλγόριθμοι}
Τι είναι;
\end{frame}

\subsubsection*{Επιλογή Ρουλέτας}

\begin{frame}
\frametitle{Επιλογή Ρουλέτας}
Τι είναι;
\end{frame}

\subsubsection*{Επιλογή Τουρνουά}

\begin{frame}
\frametitle{Επιλογή Τουρνουά}
Τι είναι;
\end{frame}

\subsubsection*{Ελιτισμός}

\begin{frame}
\frametitle{Ελιτισμός}
Τι είναι;
\end{frame}

\subsection{Αναπαραγωγή των Γονέων}

\begin{frame}
\frametitle{Αναπαραγωγή των Γονέων}
Τι είναι;
\end{frame}

\subsubsection{Διασταύρωση Ενός Σημείου}

\begin{frame}
\frametitle{Διασταύρωση Ενός Σημείου}
Τι είναι;
\end{frame}

\subsubsection{Διασταύρωση Δύο Σημείων}

\begin{frame}
\frametitle{Διασταύρωση Δύο Σημείων}
Τι είναι;
\end{frame}

\subsubsection{Μετάλλαξη}

\begin{frame}
\frametitle{Μετάλλαξη}
Τι είναι;
\end{frame}

\subsection{Παράλληλοι Γενετικοί Αλγόριθμοι}

\begin{frame}
\frametitle{Παράλληλοι Γενετικοί Αλγόριθμοι}
\begin{itemize}
  \item Coarse Grain 
  \item Fine Grain
\end{itemize}
\end{frame}

\subsection{Εξελικτικές Στρατηγικές}

\begin{frame}
\frametitle{Εξελικτικές Στρατηγικές}
\begin{itemize}
  \item Προτάθηκαν στις αρχές του '60 από τους Ingo Rechenberg και Hans-Paul Schewefel.
  \item Αφορούν την επίλυση τεχνικών προβλημάτων βελτιστοποίησης.
  \item Χρησιμοποιούν μόνο τελεστές μετάλλαξης.
  \item Στην απλή της μορφή κάθε γονέας δημιουργεί έναν απόγονο για κάθε γενιά.
\end{itemize}
\end{frame}

\subsection{Γενετικός Προγραμματισμός}

\begin{frame}
\frametitle{Γενετικός Προγραμματισμός}
\begin{itemize}
  \item Αναζήτηση του αλγορίθμου που αρμόζει στην επίλυση ενός προβλήματος.
  \item Εξέλιξη του αλγορίθμου που λύνει το πρόβλημα.
  \item Σημαντική ενίσχυση από τον John Koza το '90.  
  \item Προγράμματα == Δεδομένα.
  \item Εφαρμογή τελεστών διασταύρωσης και μετάλλαξης.
\end{itemize}
\end{frame}

%------------------------------------------------

\section{Παρουσίαση του προβλήματος}

\begin{frame}
\frametitle{Παρουσίαση του προβλήματος}
\begin{figure}[!t]
    \centering
    \begin{tikzpicture}
    	\begin{axis}[
            xlabel={x},
            ylabel={y},
            width = 0.6\columnwidth,
            %height = 0.8\columnwidth,
            colorbar,
            legend cell align = left,
            legend pos = north west,
            %colormap = {whiteblack}{gray(0cm) = (1); gray(1cm) = (0)},
            colorbar style = {title=Αριθμός Σφηκών, /tikz/.cd},
            font = \footnotesize]

            \addplot +[scatter, only marks, point meta=explicit] table [meta=wasps, col sep=comma] {../figures/waspNests.csv};
            \addlegendentry{Σφηκοφωλιά}
        \end{axis}
    \end{tikzpicture}
    \caption{Χάρτης σφηκοφωλιών}
    \label{fig_waspNestsMap}
\end{figure}
\end{frame}

%------------------------------------------------

\section{Επίλυση}

\subsection{JGAP}

\begin{frame}
\frametitle{Βήματα υλοποίησης}
\begin{enumerate}
  \item Σχεδιασμός χρωμοσώματος
  \item Υλοποίηση συνάρτησης καταλληλότητας
  \item Ορισμός των παραμέτρων
  \item Δημιουργία ενός πληθυσμού από υποψήφιες λύσεις
  \item Εξέλιξη του πληθυσμού
\end{enumerate}
\end{frame}

\subsection{Ανάλυση Προβλήματος}

\begin{frame}
\frametitle{Ανάλυση Προβλήματος}
\begin{itemize}
  \item Πρόβλημα μεγιστοποίησης -> Αύξηση αριθμού εξοντωμένων σφηκών
  \item Πρόβλημα ελαχιστοποίησης -> Μείωση του αριθμού των σφηκών που απομένουν στη σοφίτα
\end{itemize}
\end{frame}

\subsection{Περιγραφή Λύσης}

\begin{frame}
\frametitle{Περιγραφή Λύσης}
Τι είναι;
\end{frame}

\subsection{Σύγκριση Αποτελεσμάτων}

\subsubsection{Σύγκριση με τη Βέλτιστη Λύση}
\begin{frame}
\frametitle{Σύγκριση με τη Βέλτιστη Λύση}
Τι είναι;
\end{frame}

\subsubsection{Σύγκριση Μεθόδου Επιλογής}
\begin{frame}
\frametitle{Σύγκριση Μεθόδου Επιλογής}
Τι είναι;
\end{frame}

\subsubsection{Σύγκριση Μεγέθους Πληθυσμού}

\begin{frame}
\frametitle{Σύγκριση Μεγέθους Πληθυσμού}
Τι είναι;
\end{frame}

\subsubsection{Σύγκριση βάσει ποσοστού ανασυνδυασμού}

\begin{frame}
\frametitle{Σύγκριση βάσει ποσοστού ανασυνδυασμού}
Τι είναι;
\end{frame}

\section{Συμπέρασμα}

\begin{frame}
\frametitle{Συμπέρασμα}
Τι είναι;
\end{frame}


\begin{frame}

\Huge{\centerline{Thank You}}
\Large{\centerline{Questions?}}
\end{frame}

%----------------------------------------------------------------------------------------

\end{document} 