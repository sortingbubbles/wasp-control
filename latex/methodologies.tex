\section{Μοντέλα Εξελικτικών Αλγορίθμων}
Όπως αναφέρθηκε πιο πριν, οι εξελικτικοί αλγόριθμοι είναι εμπνευσμένοι από μηχανισμούς της βιολογικής εξέλιξης όπως είναι για παράδειγμα η αναπαραγωγή, η μετάλλαξη, ο ανασυνδυασμός και η φυσική επιλογή. \cite{_evolutionary_2014} Υπάρχουν διάφορα μοντέλα που μπορούν να χρησιμοποιηθούν για την επίλυση ενός προβλήματος με την χρήση εξελικτικών αλγορίθμων. Τα πιο διαδεδομένα από αυτά τα μοντέλα είναι οι Γενετικοί Αλγόριθμοι (Genetic Algorithms), ο Γενετικός Προγραμματισμός (Genetic Programming), οι Εξελικτικές Στρατηγικές (Evolution Strategies) και ο Εξελικτικός Προγραμματισμός (Evolutionary Programming). Οι τεχνικές αυτές δεν διαφέρουν πάρα πολύ μεταξύ τους, παρά μόνο στις λεπτομέρειες της υλοποίησής τους. \cite{Adamidis}

Η λειτουργία ενός εξελικτικού αλγορίθμου βασίζεται στην αρχικοποίηση του πληθυσμού και τη συνεχή εξέλιξη από γενιά σε γενιά με την λειτουργία της επιλογής, του ανασυνδυασμού και της μετάλλαξης. Αφού αρχικά γίνει η αρχικοποίηση του πληθυσμού, ακολουθεί η αξιολόγηση (evaluation) του με τη χρήση μιας συνάρτησης ποιότητας (fitness function) η οποία αποδίδει μια τιμή ποιότητας για κάθε άτομο του πληθυσμού στο συγκεκριμένο περιβάλλον.

Ειδικότερα, κατά την διάρκεια της επιλογής γίνεται η επιλογή γονέων και στη συνέχεια η επιβίωση. Στην επιλογή γονέων καθορίζεται ποια άτομα θα γίνουν γονείς και πόσους απογόνους θα αποκτήσουν. Οι απόγονοι δημιουργούνται με την ανταλλαγή πληροφοριών μεταξύ των γονέων και την περαιτέρω διαταραχή των απογόνων. Οι δύο παραπάνω διαδικασίες ονομάζονται ανασυνδυασμός και μετάλλαξη.

Στη συνέχεια αξιολογούνται οι απόγονοι με τη χρήση της συνάρτησης ποιότητας και στο τέλος επιλέγονται τα άτομα που θα επιβιώσουν και θα περάσουν στην επόμενη γενιά. Αυτή η διαδικασία ονομάζεται εξελικτικός κύκλος (evolutionary cycle).

\begin{figure}[!t]
    \centering
    \def\svgwidth{2.1in}
    \input{./figures/EvolutionaryCycle.pdf_tex}
    \caption{Εξελικτικός κύκλος}
    \label{fig_evCycl}
\end{figure}

Το μοντέλο που χρησιμοποιείται πιο συχνά για την επίλυση προβλημάτων με την χρήση εξελικτικών αλγόριθμων είναι οι γενετικοί αλγόριθμοι, οι οποίοι αναλύονται παρακάτω.

\subsection{Γενετικοί Αλγόριθμοι}

Οι γενετικοί αλγόριθμοι εφευρέθηκαν την δεκαετία του 1960 από τον John Holland, και αναπτύχθηκαν περισσότερο στο Πανεπιστήμιο του Μίσιγκαν από τον ίδιο, τους φοιτητές του και τους συναδέλφους του. Ο στόχος του Holland ήταν να μελετήσει το φαινόμενο της προσαρμογής όπως αυτό παρατηρείται στη φύση, και όχι να επιλύσει συγκεκριμένα προβλήματα. Ήθελε με αυτόν τον τρόπο να αναπτύξει μεθόδους για την εισαγωγή των μηχανισμών της φυσικής επιλογής σε υπολογιστικά συστήματα. \cite{Melanie1999}

\subsection{Τρόποι Αναπαράστασης}

Ο πιο συνηθισμένος τρόπος αναπαράστασης των χρωμοσωμάτων στους γενετικούς αλγόριθμους είναι η σταθερού-μήκους συμβολοσειρά δυαδικών ψηφίων (bit string). Η χρήση των bit string διευκολύνει την διαδικασία εύρεσης απογόνων.

\begin{figure}[!t]
    \renewcommand{\arraystretch}{1.3}
    \label{fig_bit_string}
    \centering
    \begin{tabular}{c|c|c|c|c|c|c|c}
        \hline
        \ldots & 0 & 0 & 1 & 0 & 1 & 1 & \ldots\\
        \hline
    \end{tabular}
    \caption{Παράδειγμα bit string}
\end{figure}

Ένα πρόβλημα που παρουσιάζουν τα bit string είναι ο λεγόμενος «γκρεμός Hamming». Για παράδειγμα, ο αριθμός $255$ στο δυαδικό σύστημα γράφεται ως $011111111$ και ο αριθμός $256$ ως $100000000$. Για την αλλαγή από τον έναν αριθμό στον άλλον πρέπει να αλλάξουν $9$ bits, που σημαίνει ότι είναι πολύ πιθανό να πρέπει να βρεθούν πολλοί απόγονοι μέχρι το χρωμόσωμα να φτάσει σε μια πιο βέλτιστη λύση. Αυτό το πρόβλημα μπορεί να αντιμετωπιστεί κωδικοποιώντας τους αριθμούς με τον κώδικα Gray. Ό κώδικας Gray είναι ένα δυαδικό σύστημα όπου δύο διαδοχικοί αριθμοί διαφέρουν μόνο κατά ένα bit. \cite{Lehre}

\begin{table}[!t]
    \renewcommand{\arraystretch}{1.3}
    \caption{Κωδικοποίηση Gray 3 bit}
    \label{table_gray_code}
    \centering
    \begin{tabular}{c|c|c}
        \hline
        \bfseries Δεκαδικό & \bfseries Δυαδικό & \bfseries Κώδικας Gray\\
        \hline\hline
        0 & 000 & 000\\
        1 & 001 & 001\\
        2 & 010 & 011\\
        3 & 011 & 010\\
        4 & 100 & 110\\
        5 & 101 & 111\\
        6 & 110 & 101\\
        7 & 111 & 100\\
        \hline
    \end{tabular}
\end{table}

Άλλοι τρόποι αναπαράστασης που μπορούν να χρησιμοποιηθούν ανάλογα με το πρόβλημα είναι οι μεταθέσεις και η δεντρική δομή. Οι μεταθέσεις χρησιμοποιούνται συνήθως σε προβλήματα ταξινόμησης, όπως για παράδειγμα το γνωστό πρόβλημα του περιπλανώμενου πωλητή (το κάθε χρωμόσωμα δίνει την σειρά που ο πωλητής θα επισκεφθεί τις πόλεις). \cite{Obitko}

\subsection{Δημιουργία Πληθυσμού}

Στη φάση της δημιουργίας του αρχικού πληθυσμού, συνήθως δημιουργούνται Ν τυχαία σταθερού-μήκους bit strings, όπου Ν είναι το μέγεθος του πληθυσμού που έχει οριστεί. Υπάρχουν έρευνες που αποδεικνύουν ότι όσο μεγαλύτερο το μέγεθος του πληθυσμού, τόσο μεγαλύτερη η ακρίβεια του γενετικού αλγόριθμου, αλλά και τόσες περισσότερες γενιές χρειάζονται για να αυξηθεί η σύγκλιση. \cite{Gotshall2008}

\subsection{Επιλογή Γονέων}

Αφού δημιουργηθεί ο αρχικός πληθυσμός, θα πρέπει να επιλεχθούν τα χρωμοσώματα που θα γίνουν γονείς και θα παράγουν τα χρωμοσώματα-παιδιά που θα χρησιμοποιηθούν για τον επόμενο κύκλο του γενετικού αλγόριθμου. Είναι επιθυμητό τα χρωμοσώματα που δίνουν καλύτερη λύση στο πρόβλημα να έχουν περισσότερες πιθανότητες να επιλεχθούν. Για τον έλεγχο της ποιότητας ενός χρωμοσώματος, χρησιμοποιείται η συνάρτηση καταλληλότητας (fitness function). Η συνάρτηση καταλληλότητας δέχεται σαν είσοδο ένα χρωμόσωμα, και επιστρέφει έναν αριθμό στο διάστημα [0, 1], όπου το 0 υποδηλώνει ότι το χρωμόσωμα δεν αποτελεί καλή λύση του προβλήματος, ενώ το 1 υποδηλώνει ότι το χρωμόσωμα είναι τέλειο.

Υπάρχουν διάφορες τεχνικές για την επιλογή των γονέων. Μία από αυτές τις τεχνικές είναι η επιλογή ρουλέτας (roulette wheel selection) που αναφέρθηκε παραπάνω, όπου τα χρωμοσώματα καταλαμβάνουν χώρο στη «ρουλέτα» ανάλογα με την ποιότητα τους (οπότε υπάρχει μεγαλύτερη πιθανότητα να «πέσει η μπίλια» στα καλύτερα χρωμοσώματα). Μια άλλη τεχνική ονομάζεται επιλογή τουρνουά (tournament selection), και λειτουργεί επιλέγοντας ένα τυχαίο υποσύνολο k χρωμοσωμάτων από τον αρχικό πληθυσμό, και επιλέγοντας το καλύτερο χρωμόσωμα από αυτό το υποσύνολο.

\subsection{Αναπαραγωγή των Γονέων}

Μετά την χρήση μιας από τις παραπάνω μεθόδους για την επιλογή των χρωμοσωμάτων που θα χρησιμοποιηθούν ως γονείς στην αναπαραγωγή, πρέπει να εφαρμοστεί ένας ή περισσότεροι τελεστές για την δημιουργία των χρωμοσωμάτων-παιδιών. Μερικοί από τους τελεστές που χρησιμοποιούνται συχνότερα είναι η διασταύρωση (crossover) και η μετάλλαξη.

\subsubsection{Διασταύρωση ενός σημείου}

Στη διασταύρωση ενός σημείου (one-point crossover) επιλέγεται τυχαία ένα σημείο στο χρωμόσωμα των γονέων, και τα δεδομένα μετά από αυτό το σημείο ανταλλάσσονται μεταξύ των δύο γονέων. Για παράδειγμα, έστω ότι τα χρωμοσώματα $0110100$ και $1100110$ είναι οι δύο γονείς, και το σημείο διασταύρωσης είναι το $4$. Τότε τα καινούργια χρωμοσώματα θα είναι τα $0110110$ και $1100100$.

\begin{figure}[!t]
    \centering
    \def\svgwidth{2.5in}
    \input{./figures/OnePointCrossover.pdf_tex}
    \caption{Διασταύρωση ενός σημείου \cite{opc_fig}}
    \label{fig_opc}
\end{figure}

\subsubsection{Διασταύρωση δύο σημείων}

Στη διασταύρωση δύο σημείων (two-point crossover), επιλέγονται τυχαία δύο σημεία στο χρωμόσωμα των γονέων, και τα δεδομένα που βρίσκονται ανάμεσα στα δύο σημεία ανταλλάσσονται μεταξύ των δύο γονέων. Για παράδειγμα, έστω πάλι τα χρωμοσώματα $0110100$ και $1100110$, και τα δύο τυχαία σημεία είναι το 2 και το 5. Τότε, τα παιδιά που θα παραχθούν θα είναι τα $0100100$ και $1110110$.

\begin{figure}[!t]
    \centering
    \def\svgwidth{2.5in}
    \input{./figures/TwoPointCrossover.pdf_tex}
    \caption{Διασταύρωση δύο σημείων \cite{tpc_fig}}
    \label{fig_tpc}
\end{figure}

\subsubsection{Μετάλλαξη}

Στη μετάλλαξη (mutation) επιλέγεται τυχαία ένα bit από το χρωμόσωμα, και αυτό αντιστρέφεται (δηλαδή από $0$ γίνεται $1$, και από $1$ γίνεται $0$). Αυτός ο τελεστής πρέπει να έχει μικρή πιθανότητα να εφαρμοστεί (κοντά στο $1\%$), αλλιώς υπάρχει κίνδυνος η αναζήτηση να μετατραπεί σε τυχαία αναζήτηση. \cite{zotero-PNQJ3TKI}

<ΣΧΗΜΑ mutation (?)>