\section{Μοντέλα Εξελικτικών Αλγορίθμων}

Όπως αναφέρθηκε πιο πριν, οι εξελικτικοί αλγόριθμοι είναι εμπνευσμένοι από μηχανισμούς της βιολογικής εξέλιξης όπως είναι για παράδειγμα η αναπαραγωγή, η μετάλλαξη, ο ανασυνδυασμός και η φυσική επιλογή. \cite{_evolutionary_2014} Υπάρχουν διάφορα μοντέλα που μπορούν να χρησιμοποιηθούν για την επίλυση ενός προβλήματος με την χρήση εξελικτικών αλγορίθμων. Τα πιο διαδεδομένα από αυτά τα μοντέλα είναι οι Γενετικοί Αλγόριθμοι (Genetic Algorithms), ο Γενετικός Προγραμματισμός (Genetic Programming), οι Εξελικτικές Στρατηγικές (Evolution Strategies) και ο Εξελικτικός Προγραμματισμός (Evolutionary Programming). Οι τεχνικές αυτές δεν διαφέρουν πάρα πολύ μεταξύ τους, παρά μόνο στις λεπτομέρειες της υλοποίησής τους. \cite{Adamidis}

Η λειτουργία ενός εξελικτικού αλγορίθμου βασίζεται στην αρχικοποίηση του πληθυσμού και τη συνεχή εξέλιξη από γενιά σε γενιά με την λειτουργία της επιλογής, του ανασυνδυασμού και της μετάλλαξης. Αφού αρχικά γίνει η αρχικοποίηση του πληθυσμού, ακολουθεί η αξιολόγηση (evaluation) του με τη χρήση μιας συνάρτησης ποιότητας (fitness function) η οποία αποδίδει μια τιμή ποιότητας για κάθε άτομο του πληθυσμού στο συγκεκριμένο περιβάλλον.

Ειδικότερα, κατά την διάρκεια της επιλογής γίνεται η επιλογή γονέων και στη συνέχεια η επιβίωση. Στην επιλογή γονέων καθορίζεται ποια άτομα θα γίνουν γονείς και πόσους απογόνους θα αποκτήσουν. Οι απόγονοι δημιουργούνται με την ανταλλαγή πληροφοριών μεταξύ των γονέων και την περαιτέρω διαταραχή των απογόνων. Οι δύο παραπάνω διαδικασίες ονομάζονται ανασυνδυασμός και μετάλλαξη.

Στη συνέχεια αξιολογούνται οι απόγονοι με τη χρήση της συνάρτησης ποιότητας και στο τέλος επιλέγονται τα άτομα που θα επιβιώσουν και θα περάσουν στην επόμενη γενιά. Αυτή η διαδικασία ονομάζεται εξελικτικός κύκλος (evolutionary cycle). Στον αλγόριθμο \ref{algo_evAlg} παρουσιάζεται αυτή η διαδικασία σε μορφή ψευδοκώδικα.

%\begin{figure}
%    \centering
%    \def\svgwidth{2.1in}
%    \input{./figures/EvolutionaryCycle.pdf_tex}
%    \caption{Εξελικτικός κύκλος}
%    \label{fig_evCycl}
%\end{figure}

\begin{algorithm}[!t]
    \caption{Γενική μορφή ενός εξελικτικού αλγόριθμου}
    \label{algo_evAlg}
    \KwIn{populationSize, maxEvolutions}
    \KwOut{The chromosome with the best fitness value}
    $population \gets randomPopulation(populationSize)$\;
    $evolutions \gets 0$\;
    \While{$evolutions < maxEvolutions$}{
        $parents \gets selectParents(population)$\;
        $population \gets recombine(parents)$\;
        $population \gets mutate(population)$\;
        $evolutions \gets evolutions + 1$\;
    }
    \Return{$getBestChromosome(population)$}\;
\end{algorithm}

Το μοντέλο που χρησιμοποιείται πιο συχνά για την επίλυση προβλημάτων με την χρήση εξελικτικών αλγόριθμων είναι οι γενετικοί αλγόριθμοι, οι οποίοι αναλύονται παρακάτω.

\subsection{Γενετικοί Αλγόριθμοι}

Οι γενετικοί αλγόριθμοι εφευρέθηκαν την δεκαετία του 1960 από τον John Holland, και αναπτύχθηκαν περισσότερο στο Πανεπιστήμιο του Μίσιγκαν από τον ίδιο, τους φοιτητές του και τους συναδέλφους του. Ο στόχος του Holland ήταν να μελετήσει το φαινόμενο της προσαρμογής όπως αυτό παρατηρείται στη φύση, και όχι να επιλύσει συγκεκριμένα προβλήματα. Ήθελε με αυτόν τον τρόπο να αναπτύξει μεθόδους για την εισαγωγή των μηχανισμών της φυσικής επιλογής σε υπολογιστικά συστήματα. \cite{Melanie1999}

O Holland ασχολήθηκε με αλγορίθμους που χειρίζονται συμβολοσειρές από δυαδικά ψηφία. Θεωρούσε αυτούς τους αλγόριθμους ως μια αφηρημένη μορφή της φυσικής εξέλιξης. Οι γενετικοί αλγόριθμοι του μπορούσαν να αναπαρασταθούν με μια σειρά από διαδικαστικά βήματα για τη μετάβαση από τον ένα πληθυσμό τεχνητών χρωμοσωμάτων σε έναν καινούργιο. \cite{Negnevitsky2005}

Στην ουσία, οι γενετικοί αλγόριθμοι είναι μια κλάση από στοχαστικούς αλγόριθμους αναζήτησης οι οποίοι είναι βασισμένοι στη βιολογική εξέλιξη. Ακολουθούν μια επαναληπτική διαδικασία και κάθε επανάληψη ονομάζεται γενιά.

Η ευρεία διάδοση των γενετικών αλγορίθμων μπορεί να αποδοθεί στους παρακάτω παράγοντες \cite{Mitchell1997} :

\begin{itemize}
  \item Είναι γνωστό ότι η εξέλιξη είναι μια αποδοτική μέθοδος για την προσαρμογή σε βιολογικά συστήματα και οδηγεί στην επιτυχία.
  \item Οι γενετικοί αλγόριθμοι μπορούν να αναζητήσουν την καλύτερη λύση σε συστήματα που περιέχουν περίπλοκα στοιχεία αλληλεπίδρασης, στα οποία είναι δύσκολος ο υπολογισμός της επίδρασης του κάθε στοιχείου σε ολόκληρο το σύστημα και στη συνάρτηση καταλληλότητας.
  \item Οι γενετικοί αλγόριθμοι μπορούν εύκολα να διαμορφωθούν έτσι ώστε να λειτουργούν παράλληλα και να εκμεταλλευτούν την υπολογιστική δύναμη ενός ισχυρού υπολογιστή.
\end{itemize}

Στη φύση υπάρχει η δυνατότητα της προσαρμογής και μάθησης χωρίς την καθοδήγηση από κάποιον τρίτο. Το ίδιο ισχύει και για τους γενετικούς αλγορίθμους. Οι μηχανισμοί που συνδέουν τους γενετικούς αλγορίθμους με το πρόβλημα είναι η αναπαράσταση και η αξιολόγηση.

\subsection{Τρόποι Αναπαράστασης}

Ο πιο συνηθισμένος τρόπος αναπαράστασης των χρωμοσωμάτων στους γενετικούς αλγόριθμους είναι η σταθερού-μήκους συμβολοσειρά δυαδικών ψηφίων (bit string). Η χρήση των bit string διευκολύνει την διαδικασία εύρεσης απογόνων.

\begin{figure}[!t]
    \renewcommand{\arraystretch}{1.3}
    \label{fig_bit_string}
    \centering
    \begin{tabular}{c|c|c|c|c|c|c|c}
        \hline
        \ldots & 0 & 0 & 1 & 0 & 1 & 1 & \ldots\\
        \hline
    \end{tabular}
    \caption{Παράδειγμα bit string}
\end{figure}

Ένα πρόβλημα που παρουσιάζουν τα bit string είναι ο λεγόμενος «γκρεμός Hamming». Για παράδειγμα, ο αριθμός $255$ στο δυαδικό σύστημα γράφεται ως $011111111$ και ο αριθμός $256$ ως $100000000$. Για την αλλαγή από τον έναν αριθμό στον άλλον πρέπει να αλλάξουν $9$ bits, που σημαίνει ότι είναι πολύ πιθανό να πρέπει να βρεθούν πολλοί απόγονοι μέχρι το χρωμόσωμα να φτάσει σε μια πιο βέλτιστη λύση. Αυτό το πρόβλημα μπορεί να αντιμετωπιστεί κωδικοποιώντας τους αριθμούς με τον κώδικα Gray. Ό κώδικας Gray είναι ένα δυαδικό σύστημα όπου δύο διαδοχικοί αριθμοί διαφέρουν μόνο κατά ένα bit. \cite{Lehre} Στον πίνακα \ref{table_gray_code} φαίνεται η κωδικοποίηση Gray για ένα bit string μήκους 3 bit.

\begin{table}[!t]
    \renewcommand{\arraystretch}{1.3}
    \caption{Κωδικοποίηση Gray 3 bit}
    \label{table_gray_code}
    \centering
    \begin{tabular}{c|c|c}
        \hline
        \bfseries Δεκαδικό & \bfseries Δυαδικό & \bfseries Κώδικας Gray\\
        \hline\hline
        0 & 000 & 000\\
        1 & 001 & 001\\
        2 & 010 & 011\\
        3 & 011 & 010\\
        4 & 100 & 110\\
        5 & 101 & 111\\
        6 & 110 & 101\\
        7 & 111 & 100\\
        \hline
    \end{tabular}
\end{table}

Άλλοι τρόποι αναπαράστασης που μπορούν να χρησιμοποιηθούν ανάλογα με το πρόβλημα είναι οι μεταθέσεις και η δεντρική δομή. Οι μεταθέσεις χρησιμοποιούνται συνήθως σε προβλήματα ταξινόμησης, όπως για παράδειγμα το γνωστό πρόβλημα του περιπλανώμενου πωλητή (το κάθε χρωμόσωμα δίνει την σειρά που ο πωλητής θα επισκεφθεί τις πόλεις). \cite{Obitko}

\subsection{Δημιουργία Πληθυσμού}

Στη φάση της δημιουργίας του αρχικού πληθυσμού, συνήθως δημιουργούνται $N$ τυχαία σταθερού-μήκους bit strings, όπου $N$ είναι το μέγεθος του πληθυσμού που έχει οριστεί. Υπάρχουν έρευνες που αποδεικνύουν ότι όσο μεγαλύτερο το μέγεθος του πληθυσμού, τόσο μεγαλύτερη η ακρίβεια του γενετικού αλγόριθμου, αλλά και τόσες περισσότερες γενιές χρειάζονται για να αυξηθεί η σύγκλιση. \cite{Gotshall2008}

\subsection{Αξιολόγηση Πληθυσμού}

%Αφού δημιουργηθεί ο αρχικός πληθυσμός, θα πρέπει να επιλεχθούν τα χρωμοσώματα που θα γίνουν γονείς και θα παράγουν τα χρωμοσώματα-παιδιά που θα χρησιμοποιηθούν για τον επόμενο κύκλο του γενετικού αλγόριθμου. Είναι επιθυμητό τα χρωμοσώματα που δίνουν καλύτερη λύση στο πρόβλημα να έχουν περισσότερες πιθανότητες να επιλεχθούν.

Αφού δημιουργηθεί ο αρχικός πληθυσμός, θα πρέπει να αξιολογηθεί το πόσο καλή είναι η λύση που δίνει το κάθε χρωμόσωμα. Για τον έλεγχο της ποιότητας ενός χρωμοσώματος, χρησιμοποιείται η λεγόμενη συνάρτηση καταλληλότητας (fitness function) ή αλλιώς συνάρτηση αξιολόγησης.

Η συνάρτηση καταλληλότητας δέχεται σαν είσοδο ένα χρωμόσωμα, και επιστρέφει έναν αριθμό ο οποίος υποδηλώνει το πόσο καλό είναι το χρωμόσωμα. Εάν επιλύεται πρόβλημα μεγιστοποίησης, τότε όσο πιο μεγάλη τιμή επιστρέφει η συνάρτηση καταλληλότητας για το χρωμόσωμα, τόσο πιο καλή λύση αναπαριστά το χρωμόσωμα. Αντιθέτως, εάν επιλύεται πρόβλημα ελαχιστοποίησης, τότε όσο πιο χαμηλή τιμή καταλληλότητας έχει ένα χρωμόσωμα, τόσο καλύτερη λύση αναπαριστά.

\subsection{Επιλογή Γονέων}

Μετά την αξιολόγηση του πληθυσμού, πρέπει να επιλεχθούν τα χρωμοσώματα που θα γίνουν γονείς και θα παράγουν τα χρωμοσώματα-παιδιά που θα χρησιμοποιηθούν για τον επόμενο κύκλο του γενετικού αλγόριθμου. Είναι επιθυμητό τα χρωμοσώματα που δίνουν καλύτερη λύση στο πρόβλημα να έχουν περισσότερες πιθανότητες να επιλεχθούν. Υπάρχουν διάφορες τεχνικές για την επιλογή των γονέων. Οι πιο διαδεδομένες από αυτές αναλύονται παρακάτω.

\subsubsection{Επιλογή Ρουλέτας}

Μία από αυτές τις τεχνικές είναι η επιλογή ρουλέτας (roulette wheel selection) που αναφέρθηκε παραπάνω, όπου τα χρωμοσώματα καταλαμβάνουν χώρο στη «ρουλέτα» ανάλογα με την ποιότητα τους (οπότε υπάρχει μεγαλύτερη πιθανότητα να «πέσει η μπίλια» στα καλύτερα χρωμοσώματα).

\begin{figure*}[!t]
    \centering
    \begin{subtable}[t!]{\linewidth}
        \centering
        \begin{tabular}{@{}crrrr@{}}
            \toprule
            Α/Α Χρωμοσώματος & Καταλληλότητα & Συνολικό ποσοστό & Κατάταξη & Ποσοστό κατάταξης \\ \midrule
            1                & 40            & 14.8\%           & 4        & 19\%              \\
            2                & 110           & 40.7\%           & 6        & 28.5\%            \\
            3                & 30            & 11.1\%           & 3        & 14.3\%            \\
            4                & 25            & 9.3\%            & 2        & 9.6\%             \\
            5                & 45            & 16.7\%           & 5        & 23.8\%            \\
            6                & 20            & 7.4\%            & 1        & 4.8\%             \\ \midrule
            Σύνολο           & 270           & 100\%            &          & 100\%             \\ \bottomrule
        \end{tabular}
        \caption{Πληθυσμός}
    \end{subtable}
    \\
    \begin{subfigure}[t!]{.4\linewidth}
        \centering
        \begin{tikzpicture}
            \pie[rotate=45, radius=3]{14.8/1, 7.4/6, 16.7/5, 9.3/4, 11.1/3, 40.7/2}
        \end{tikzpicture}
        \caption{Ποσοστά καταλληλότητας}
    \end{subfigure}
    ~
    \begin{subfigure}[t!]{.4\linewidth}
        \centering
        \begin{tikzpicture}
            \pie[rotate=45, radius=3]{19/1, 4.8/6, 23.8/5, 9.6/4, 14.3/3, 28.5/2}
        \end{tikzpicture}
        \caption{Ποσοστά κατάταξης}
    \end{subfigure}

    \caption{Παράδειγμα επιλογής ρουλέτας}
    \label{fig_rouletteWheel}
\end{figure*}

\begin{algorithm}[!t]
    \caption{Ψευδοκώδικας επιλογής ρουλέτας}
    \label{algo_rouletteWheel}
    \KwIn{population}
    \KwOut{The selected chromosomes}
    $fitnessSum \gets 0$\;
    \ForEach{$x$ in $population$}{
        $fitnessSum \gets fitnessSum + fitness(x)$\;
    }
    \While{$newPopulation$ is not full}{
        $randNum \gets randBetween(0, fitnessSum)$\;
        $relFitness \gets 0$\;
        $i \gets 0$\;
        \While{$randNum > relFitness$}{
            $relFit \gets relFit + fitness(population_i)$\;
            $i \gets i + 1$\;
        }
        $newPopulation.add(population_{i-1})$\;
    }
    \Return{$newPopulation$}\;
\end{algorithm}

\subsubsection{Επιλογή Κατάταξης}

Η επιλογή κατάταξης (rank selec\-tion) μοιάζει πολύ με την επιλογή ρουλέτας, μόνο που η πιθανότητα επιλογής ενός χρωμοσώματος είναι ανάλογη της σχετικής καταλληλότητας (κατάταξη) αντί τις απόλυτης καταλληλότητας. Για έναν πληθυσμό με $N$ άτομα, το καλύτερο άτομο έχει κατάταξη $N$, το επόμενο καλύτερο έχει κατάταξη $N-1$, κ.ο.κ. \cite{manlio2014}

Αυτό σημαίνει ότι δεν έχει σημασία το πόσο πολύ διαφέρει η καταλληλότητα ανάμεσα σε δύο άτομα, αλλά το πόσο διαφέρει η κατάταξή τους. Για παράδειγμα, στο σχήμα \ref{fig_rouletteWheel} φαίνεται ότι ενώ το χρωμόσωμα 2 έχει ποσοστό $40.7\%$ να επιλεγεί με την μέθοδο επιλογής ρουλέτας, στην μέθοδο επιλογής κατάταξης έχει μόνο $28.5\%$ πιθανότητα να επιλεχτεί.

\subsubsection{Επιλογή Τουρνουά}

Μια άλλη τεχνική ονομάζεται επιλογή τουρνουά (tournament selection), και λειτουργεί επιλέγοντας ένα τυχαίο υποσύνολο k χρωμοσωμάτων από τον αρχικό πληθυσμό, και επιλέγοντας το καλύτερο χρωμόσωμα από αυτό το υποσύνολο.

\subsection{Αναπαραγωγή των Γονέων}

Μετά την χρήση μιας από τις παραπάνω μεθόδους για την επιλογή των χρωμοσωμάτων που θα χρησιμοποιηθούν ως γονείς στην αναπαραγωγή, πρέπει να εφαρμοστεί ένας ή περισσότεροι τελεστές για την δημιουργία των χρωμοσωμάτων-παιδιών. Μερικοί από τους τελεστές που χρησιμοποιούνται συχνότερα είναι η διασταύρωση (crossover) και η μετάλλαξη.

\subsubsection{Διασταύρωση ενός σημείου}

Στη διασταύρωση ενός σημείου (one-point crossover) επιλέγεται τυχαία ένα σημείο στο χρωμόσωμα των γονέων, και τα δεδομένα μετά από αυτό το σημείο ανταλλάσσονται μεταξύ των δύο γονέων (σχήμα \ref{fig_opc}). Για παράδειγμα, έστω ότι τα χρωμοσώματα $0110100$ και $1100110$ είναι οι δύο γονείς, και το σημείο διασταύρωσης είναι το $4$. Τότε τα καινούργια χρωμοσώματα θα είναι τα $0110110$ και $1100100$.

\begin{figure}[!t]
    \centering
    \def\svgwidth{2.5in}
    \input{./figures/OnePointCrossover.pdf_tex}
    \caption{Διασταύρωση ενός σημείου \cite{opc_fig}}
    \label{fig_opc}
\end{figure}

\subsubsection{Διασταύρωση δύο σημείων}

Στη διασταύρωση δύο σημείων (two-point crossover), επιλέγονται τυχαία δύο σημεία στο χρωμόσωμα των γονέων, και τα δεδομένα που βρίσκονται ανάμεσα στα δύο σημεία ανταλλάσσονται μεταξύ των δύο γονέων (σχήμα \ref{fig_tpc}). Για παράδειγμα, έστω πάλι τα χρωμοσώματα $0110100$ και $1100110$, και τα δύο τυχαία σημεία είναι το 2 και το 5. Τότε, τα παιδιά που θα παραχθούν θα είναι τα $0100100$ και $1110110$.

\begin{figure}[!t]
    \centering
    \def\svgwidth{2.5in}
    \input{./figures/TwoPointCrossover.pdf_tex}
    \caption{Διασταύρωση δύο σημείων \cite{tpc_fig}}
    \label{fig_tpc}
\end{figure}

\subsubsection{Μετάλλαξη}

Στη μετάλλαξη (mutation) επιλέγεται τυχαία ένα bit από το χρωμόσωμα, και αυτό αντιστρέφεται (δηλαδή από $0$ γίνεται $1$, και από $1$ γίνεται $0$, σχήμα \ref{fig_mutation}). Αυτός ο τελεστής πρέπει να έχει μικρή πιθανότητα να εφαρμοστεί (κοντά στο $1\%$), αλλιώς υπάρχει κίνδυνος η αναζήτηση να μετατραπεί σε τυχαία αναζήτηση. \cite{zotero-PNQJ3TKI}

\begin{figure}[!t]
    \centering
    \begin{subtable}[b]{.4\linewidth}
        \centering
        \begin{tabular}{|c|c|c|c|}
            \hline
            1 & 0 & \cellcolor{gray!25}0 & 0\\
            \hline
        \end{tabular}
        \caption{Πριν}
        \label{fig_beforeMutation}
    \end{subtable}
    ~
    \begin{subtable}[b]{.3\linewidth}
        \centering
        \begin{tabular}{|c|c|c|c|}
            \hline
            1 & 0 & \cellcolor{gray!25}1 & 0\\
            \hline
        \end{tabular}
        \caption{Μετά}
        \label{fig_afterMutation}
    \end{subtable}
    \caption{Μετάλλαξη ενός χρωμοσώματος}
    \label{fig_mutation}
\end{figure}

\subsection{Παράλληλοι Γενετικοί Αλγόριθμοι}

Όπως έχει ήδη αναφερθεί, ένα από τα πλεονεκτήματα των γενετικών αλγορίθμων είναι η δυνατότητα της παράλληλης εφαρμογής τους και για να επιτευχθεί αυτό, έχουν αναπτυχθεί διάφοροι τρόποι.

Μια μέθοδος (coarse grain) είναι η υποδιαίρεση του πληθυσμού σε ξεχωριστές ομάδες ατόμων, οι οποίες ονομάζονται δήμοι (demes). Κάθε δήμος εκχωρείται σε έναν διαφορετικό υπολογιστικό κόμβο και πραγματοποιείται η κανονική λειτουργία του γενετικού αλγορίθμου σε κάθε κόμβο. \cite{Oravec2009}

Ανάμεσα στους δήμους πραγματοποιείται η επικοινωνία και η αναπαραγωγή με τη διαδικασία της μετανάστευσης, αλλά σε μικρότερο βαθμό από ότι μέσα στον ίδιο δήμο. Αυτή η διαδικασία έχει μοντελοποιηθεί από την ίδια τη φύση, καθώς είναι γνωστό ότι η διασταύρωση ανάμεσα σε διαχωρισμένους πληθυσμούς βιολογικών ειδών είναι ένα φαινόμενο που παρατηρείται στο βιολογικό περιβάλλον.

Το όφελος αυτής της προσέγγισης είναι ότι μειώνει το πρόβλημα του συνωστισμού των πληθυσμών, το οποίο παρατηρείται αρκετά συχνά στους μη-παράλληλους γενετικούς αλγορίθμους. Αυτό συμβαίνει επειδή το σύστημα φτάνει σε ένα τοπικό βέλτιστο, λόγω της πρόωρης εμφάνισης ενός γονότυπου που κυριαρχεί ολόκληρο το πληθυσμό.

Σε αντίθεση με τη παραπάνω μέθοδο παράλληλης εφαρμογής των γενετικών αλγορίθμων, υπάρχει μια άλλη κατά την οποία κάθε άτομο του πληθυσμού εκχωρείται σε έναν υπολογιστικό κόμβο. Με αυτόν τον τρόπο ο ανασυνδυασμός εφαρμόζεται σε γειτονικά άτομα. Έχουν προταθεί πολλοί διαφορετικοί τύποι γειτονιών, από επίπεδα δίκτυα μέχρι και σπείρες.\cite{Mitchell1997}

\subsection{Εξελικτικές Στρατηγικές}
Μια άλλη προσέγγιση για τη προσομοίωση της φυσικής εξέλιξης προτάθηκε στη Γερμανία στις αρχές τις δεκαετίας του '60. Σε αντίθεση με τους γενετικούς αλγόριθμους, αυτή η μέθοδος η οποία ονομάστηκε εξελικτικές στρατηγικές, σχεδιάστηκε για την επίλυση τεχνικών προβλημάτων βελτιστοποίησης.

Το 1963 δύο φοιτητές του Τεχνικού Πανεπιστημίου του Βερολίνου, ο Ingo Rechenberg και ο Hans-Paul Schwefel, αναζητούσαν τα καταλληλότερα σχήματα σωμάτων μέσα από τα οποία θα υπάρχει ροή νερού ή αέρα. Για την μελέτη τους χρησιμοποίησαν μια αεροδυναμική σήραγγα. Επειδή η διαδικασία της εξαγωγής των πειραμάτων ήταν επίπονη, αποφάσισαν να εφαρμόσουν τυχαίες αλλαγές στις παραμέτρους που καθόριζαν το σχήμα του σώματος, ακολουθώντας το παράδειγμα της φυσικής μετάλλαξης. Ως αποτέλεσμα, γεννήθηκαν οι εξελικτικές στρατηγικές.

Οι εξελικτικές στρατηγικές αναπτύχθηκαν ως μια εναλλακτική της διαισθητικής ικανότητας των μηχανικών. Μέχρι πρόσφατα, η χρήση τους αφορούσε τα τεχνικά προβλήματα βελτιστοποίησης που δεν διέθεταν κάποια αναλυτική αντικειμενική συνάρτηση και ούτε κάποια συμβατική μέθοδος βελτιστοποίησης, με αποτέλεσμα οι μηχανικοί να αναγκάζονται να ακολουθήσουν τη διαίσθησή τους.

Σε αντίθεση με τους γενετικούς αλγορίθμους, οι εξελικτικές στρατηγικές χρησιμοποιούν μόνο τελεστές μετάλλαξης και δεν υπάρχει η ανάγκη για την κωδικοποίηση του προβλήματος.

Στην απλή της μορφή, η οποία ονομάζεται $(1+1)$-εξελικτική στρατηγική, κάθε γονέας δημιουργεί έναν απόγονο για κάθε γενιά με την εφαρμογή της κανονικά κατανεμημένης μετάλλαξης.
\subsection{Γενετικός Προγραμματισμός}

Ένα από τα κυριότερα προβλήματα στην επιστήμη των υπολογιστών είναι η δημιουργία υπολογιστών οι οποίοι θα είναι σε θέση να λύσουν προβλήματα για τα οποία όμως δεν έχουν προγραμματισθεί ρητά. Ο γενετικός προγραμματισμός προσφέρει μια λύση σε αυτό το πρόβλημα μέσα από την εξέλιξη των υπολογιστικών προγραμμάτων χρησιμοποιώντας μεθόδους της φυσικής εξέλιξης.

Στη πραγματικότητα, ο γενετικός προγραμματισμός είναι μια επέκταση του συμβατικού γενετικού αλγορίθμου, όμως αυτή τη φορά ο στόχος δεν είναι μόνο η εξέλιξη μια δυαδικής συμβολοσειράς κάποιου προβλήματος, αλλά η εξέλιξη του ίδιου του αλγορίθμου που λύνει το πρόβλημα.

Ο γενετικός προγραμματισμός είναι μια πρόσφατη ανάπτυξη στο τομέα των εξελικτικών αλγορίθμων. Ενισχύθηκε σημαντικά στη δεκαετία του '90 από τον John Koza.
Σύμφωνα με αυτόν, ο γενετικός προγραμματισμός αναζητά σε ένα χώρο από πιθανά προγράμματα εκείνο το πρόγραμμα που αρμόζει περισσότερο για την λύση ενός συγκεκριμένου προβλήματος.

Κάθε πρόγραμμα είναι μια ακολουθία από λειτουργίες εφαρμοσμένες σε παραμέτρους. Διαφορετικές γλώσσες προγραμματισμού μπορεί να περιέχουν διαφορετικούς τύπους, τελεστές και συντακτικούς περιορισμούς. Εφόσον ο γενετικός προγραμματισμός παραποιεί τα προγράμματα με την εφαρμογή γενετικών τελεστών, η γλώσσα προγραμματισμού που χρησιμοποιείται θα πρέπει να χειρίζεται το πρόγραμμα σαν δεδομένα και όλα τα δεδομένα που παράγονται θα πρέπει να έχουν την δυνατότητα να εκτελεστούν. Για αυτούς τους λόγους επιλέχθηκε η LISP ως κύρια γλώσσα προγραμματισμού για τους γενετικούς αλγορίθμους.

Πριν την εφαρμογή του γενετικού προγραμματισμού, θα πρέπει να πραγματοποιηθούν τα εξής προκαταρκτικά βήματα:

\begin{enumerate}
  \item Καθορισμός του αριθμού των τερματικών
  \item Επιλογή του συνόλου των παραγόντων συναρτήσεων
  \item Καθορισμός της συνάρτησης καταλληλότητας
  \item Καθορισμός των παραμέτρων για τον έλεγχο της εκτέλεσης
  \item Επιλογής της μεθόδου εξαγωγής αποτελέσματος για την εκτέλεση
\end{enumerate}

Μόλις ολοκληρωθούν τα 5 παραπάνω βήματα, μπορεί να αρχίσει η εκτέλεση, η οποία ξεκινά με τη δημιουργία τυχαίου αρχικού πληθυσμού από προγράμματα. Κάθε πρόγραμμα αποτελείται από μεθόδους και τερματικά.

Στον αρχικό πληθυσμό, όλα τα προγράμματα συνήθως έχουν χαμηλή ποιότητα, αλλά κάποια μεμονωμένα είναι πιο ποιοτικά από άλλα. Ακριβώς όπως ένα ποιοτικότερο χρωμόσωμα έχει μεγαλύτερη πιθανότητα να επιλεγεί για να αναπαράγει απογόνους, έτσι και ένα ποιοτικότερο πρόγραμμα, με βάση τη συνάρτηση καταλληλότητας, έχει μεγαλύτερη πιθανότητα να επιβιώσει αντιγράφοντας τον εαυτό του στην επόμενη γενιά.

Στον γενετικό προγραμματισμό, ο τελεστής της διασταύρωσης εφαρμόζεται σε δύο πρόγραμμα τα οποία επιλέγονται με βάση την καταλληλότητα τους. Τα προγράμματα μπορούν να έχουν διαφορετικό μέγεθος και σχήμα. Τα δύο προγράμματα που παράγονται ως απόγονοι, συνθέτονται με τον ανασυνδυασμό τυχαίων σημείων των προγόνων τους. Ο τελεστής παράγει έγκυρα προγράμματα ως απογόνους ανεξαρτήτως την επιλογή των σημείων διασταύρωσης.

Ο τελεστής μετάλλαξης μπορεί τυχαία να αλλάξει οποιαδήποτε μέθοδο ή οποιοδήποτε τερματικό σε ένα πρόγραμμα. Μια μέθοδος όμως μπορεί να αντικατασταθεί μόνο από μία άλλη μέθοδο και ένα τερματικό από ένα άλλο τερματικό.

Ο γενετικός προγραμματισμός εφαρμόζει την ίδια εξελικτική προσέγγιση με τους γενετικούς αλγορίθμους. Παρόλα αυτά, ο γενετικός προγραμματισμός δεν αναπαράγει συμβολοσειρές από δυαδικά ψηφία που αντιπροσωπεύουν κωδικοποιημένες λύσεις, αλλά αναπαράγει προγράμματα υπολογιστών τα οποία λύνουν ένα συγκεκριμένο πρόβλημα.
