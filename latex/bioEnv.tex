\section{Βιολογικό Υπόβαθρο}

\subsection{Ο Κάρολος και η Θεωρία της Εξέλιξης}
Όπως οι περισσότερες ανακαλύψεις τον ανθρώπων, έτσι και οι εξελικτικοί αλγόριθμοι είναι εμπνευσμένοι από την ίδια τη φύση. Η βάση για τη δημιουργία των εξελικτικών αλγορίθμων είναι η θεωρία της βιολογικής εξέλιξης η οποία προτάθηκε από τον Κάρολο Δαρβίνο στο πολύφημο έργο του "On the Origin of Species by Means of Natural Selection, or The Preservation of Favoured Races in the Struggle for Life".

Ο Δαρβίνος αναλύει την προσαρμοστικότητα των ζώων χρησιμοποιώντας την αρχή της φυσικής επιλογής. Δηλαδή στην ικανότητα των οργανισμών να προσαρμόζονται στο περιβάλλον και να αποκτούν απογόνους. Οι απόγονοι θα έχουν τα ίδια ή παρόμοια χαρακτηριστικά με τους προγόνους τους, έτσι ώστε να μπορέσουν να επιβιώσουν και να συνεχίσουν την αναπαραγωγή του είδους τους. Αυτός είναι ένας από τους κυριότερους μηχανισμούς της φυσικής εξέλιξης.

Η θεωρία του Δαρβίνου βασίζεται σε δεδομένα και στα συμπεράσματα που προκύπτουν από αυτά τα οποία συνόψισε ο βιολόγος Ernst Mayr \cite{Mayr1982} ως εξής:
\begin{itemize}
    \item Κάθε είδος είναι γόνιμο σε τέτοιο βαθμό έτσι ώστε εάν επιβιώσουν όλοι οι απόγονοι τότε ο πληθυσμός τους θα αυξηθεί.
    \item Παρά τις περιοδικές διακυμάνσεις, οι πληθυσμοί εξακολουθούν να έχουν το ίδιο περίπου μέγεθος.
    \item Οι πόροι, όπως τα τρόφιμα και το νερό είναι περιορισμένοι και παραμένουν σταθεροί με την πάροδο του χρόνου. Έτσι προκύπτει ένας αγώνας για την επιβίωση.
    \item Τα άτομα σε έναν πληθυσμό ποικίλλουν σημαντικά το ένα από το άλλο.
    \item Τα άτομα που είναι λιγότερο κατάλληλα για το περιβάλλον έχουν τη μικρότερη πιθανότητα να επιζήσουν και να αναπαραχθούν, ενώ τα άτομα που έχουν όλα τα κατάλληλα χαρακτηριστικά για να επιβιώσουν στο περιβάλλον έχουν την μεγαλύτερη πιθανότητα να επιζήσουν και να αναπαραγάγουν απογόνους οι οποίοι κληρονομούν τα χαρακτηριστικά τους και έτσι δημιουργείται η διαδικασία της φυσικής επιλογής.
    \item Αυτή η σταδιακά αναπτυσσόμενη διαδικασία οδηγεί στην αλλαγή των πληθυσμών για την καλύτερη προσαρμοστικότητα και τελικά με τη συσσώρευση των διακυμάνσεων, καταλήγει στον σχηματισμό νέων ειδών.
\end{itemize}
Επιπρόσθετα, στο έργο του Δαρβίνου αναφέρονται οι διαφορές στα φυσικά και πνευματικά χαρακτηριστικά των οργανισμών \cite{Adamidis} όπως το το ύψος, το βάρος, η ευφυΐα, το χρώμα των ματιών και του τριχώματος. Δηλαδή αναλύονται οι αποκλίσεις ανάμεσα στους φαινότυπους των οργανισμών, οι οποίοι καθορίζουν τον τρόπο ανταπόκρισης και φυσικής ενσάρκωσης των γονέων και των παιδιών τους.

Όλες αυτές οι μεταλλάξεις είναι σημαντικές μόνο εάν δίνουν τη δυνατότητα στον οργανισμό να επιβιώσει σε συγκεκριμένες συνθήκες του περιβάλλοντος. Σε διαφορετική περίπτωση, οι μεταλλάξεις δεν έχουν καμία αξία αφού στο τέλος ο οργανισμός δεν έχει επιβιώνει και καταλήγει να εξαλείφεται.

Στη περίπτωση όμως που οι συνθήκες του περιβάλλοντος είναι ευνοϊκές και οι οργανισμοί συνεχίζουν να αναπαράγονται, ο μόνος περιορισμός που υπάρχει είναι οι διαθεσιμότητα των πόρων. Για αυτό το λόγο, όταν οι πόροι είναι ανεπαρκής, επιβιώνουν μόνο οι οργανισμοί που τους εκμεταλλεύονται πιο αποδοτικά και αποτελεσματικά.

\subsection{Ανάλυση Βιολογικών Χαρακτηριστικών}
Από τη στιγμή που οι εξελικτικοί αλγόριθμοι βασίζονται σε βιολογικά χαρακτηριστικά της φύσης, είναι φανερό ότι για να τους κατανοήσει κανείς χρειάζεται να γνωρίζει κάποια βασικά βιολογικά στοιχεία.

Το σώμα όλων των οργανισμών αποτελείται από εκατομμύρια κύτταρα. Τα περισσότερα κύτταρα περιέχουν μια ολοκληρωμένη σειρά από γονίδια. Διαθέτουν λοιπόν χιλιάδες γονίδια. Τα γονίδια περιέχουν οδηγίες που ελέγχουν την ανάπτυξή και τον τρόπο που δουλεύει το σώμα. Είναι επίσης υπεύθυνα για πολλά φυσικά χαρακτηριστικά, όπως το χρώμα των ματιών, την ομάδα αίματος και το ύψος. Οι πιθανές τιμές κάθε γονιδίου ονομάζονται τιμές χαρακτηριστικών (alleles).

Στους βιολογικούς οργανισμούς \cite{Vlaxavas}, ένα χρωμόσωμα είναι ένα μεγάλο μόριο (ακολουθία) DNA και περιέχει έναν αριθμό γονιδίων. Κάθε γονίδιο έχει μια συγκεκριμένη θέση (locus) μέσα στο χρωμόσωμα. Στο πραγματικό DNA, το αλφάβητο έχει μήκος τέσσερα και αποτελείται από τα γράμματα A, G, T και C που αντιστοιχούν στα τέσσερα διαφορετικά νουκλεοτίδια (βάσεις) που το συνθέτουν (Adenine, Guanine, Thymine και Cytosine). Ένας οργανισμός μπορεί να έχει ένα ή περισσότερα χρωμοσώματα ενώ σε κάποιους οργανισμούς κάθε κύτταρο περιέχει δύο αντίγραφα για κάθε χρωμόσωμα. Για παράδειγμα, ο άνθρωπος έχει 23 ζεύγη χρωμοσωμάτων.

Στην κλασσική προσέγγιση των γενετικών αλγορίθμων, κάθε υποψήφια λύση αναπαριστάται με μία συμβολοσειρά (string) ενός πεπερασμένου αλφαβήτου. Συνήθως χρησιμοποιείται το δυαδικό αλφάβητο, οπότε οι συμβολοσειρές ονομάζονται και δυαδικές συμβολοσειρές (bit-strings). Ωστόσο, υπάρχουν περιπτώσεις γενετικών αλγορίθμων που χρησιμοποιούν πιο σύνθετες μορφές αναπαράστασης. Στα περισσότερα προβλήματα οι λύσεις περιγράφονται με μεταβλητές διαφόρων τύπων δεδομένων, επομένως η διαδικασία της κωδικοποίησης περιλαμβάνει τη μετατροπή των τιμών αυτών των μεταβλητών στις αντίστοιχες δυαδικές.

Κάθε γενετικός αλγόριθμος έχει ένα χρωμόσωμα, δηλαδή μια συμβολοσειρά με πεπερασμένο αριθμό χαρακτήρων. Τα επιμέρους τμήματα της συμβολοσειράς που κωδικοποιούν κάποιο χαρακτηριστικό, δηλαδή κάποια μεταβλητή, αντιπροσωπεύουν τα γονίδια.

Στη γενετική, το σύνολο των παραμέτρων που αναπαρίστανται από ένα συγκεκριμένο γονίδιο που μας ενδιαφέρει ή έναν αριθμό γονιδίων αναφέρεται σαν γονότυπος (genotype). Η συνολική εμφάνιση ενός οργανισμού ή εκδήλωση ενός χαρακτηριστικού ονομάζεται φαινότυπος και εξαρτάται άμεσα από το γονότυπο του.

Κάθε γονότυπος θα μπορούσε να αποτελεί μια πιθανή λύση στο πρόβλημα, η σημασία της οποίας καθορίζεται από το χρήστη. 