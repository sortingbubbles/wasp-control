\documentclass[conference]{IEEEtran}

% *** GRAPHICS RELATED PACKAGES ***
%
\ifCLASSINFOpdf
  % \usepackage[pdftex]{graphicx}
  % declare the path(s) where your graphic files are
  % \graphicspath{{../pdf/}{../jpeg/}}
  % and their extensions so you won't have to specify these with
  % every instance of \includegraphics
  % \DeclareGraphicsExtensions{.pdf,.jpeg,.png}
\else
  % or other class option (dvipsone, dvipdf, if not using dvips). graphicx
  % will default to the driver specified in the system graphics.cfg if no
  % driver is specified.
  % \usepackage[dvips]{graphicx}
  % declare the path(s) where your graphic files are
  % \graphicspath{{../eps/}}
  % and their extensions so you won't have to specify these with
  % every instance of \includegraphics
  % \DeclareGraphicsExtensions{.eps}
\fi
% graphicx was written by David Carlisle and Sebastian Rahtz. It is
% required if you want graphics, photos, etc. graphicx.sty is already
% installed on most LaTeX systems. The latest version and documentation
% can be obtained at:
% http://www.ctan.org/tex-archive/macros/latex/required/graphics/
% Another good source of documentation is "Using Imported Graphics in
% LaTeX2e" by Keith Reckdahl which can be found at:
% http://www.ctan.org/tex-archive/info/epslatex/
%
% latex, and pdflatex in dvi mode, support graphics in encapsulated
% postscript (.eps) format. pdflatex in pdf mode supports graphics
% in .pdf, .jpeg, .png and .mps (metapost) formats. Users should ensure
% that all non-photo figures use a vector format (.eps, .pdf, .mps) and
% not a bitmapped formats (.jpeg, .png). IEEE frowns on bitmapped formats
% which can result in "jaggedy"/blurry rendering of lines and letters as
% well as large increases in file sizes.
%
% You can find documentation about the pdfTeX application at:
% http://www.tug.org/applications/pdftex

% *** SPECIALIZED LIST PACKAGES ***
%
%\usepackage{algorithmic}
% algorithmic.sty was written by Peter Williams and Rogerio Brito.
% This package provides an algorithmic environment fo describing algorithms.
% You can use the algorithmic environment in-text or within a figure
% environment to provide for a floating algorithm. Do NOT use the algorithm
% floating environment provided by algorithm.sty (by the same authors) or
% algorithm2e.sty (by Christophe Fiorio) as IEEE does not use dedicated
% algorithm float types and packages that provide these will not provide
% correct IEEE style captions. The latest version and documentation of
% algorithmic.sty can be obtained at:
% http://www.ctan.org/tex-archive/macros/latex/contrib/algorithms/
% There is also a support site at:
% http://algorithms.berlios.de/index.html
% Also of interest may be the (relatively newer and more customizable)
% algorithmicx.sty package by Szasz Janos:
% http://www.ctan.org/tex-archive/macros/latex/contrib/algorithmicx/

% *** ALIGNMENT PACKAGES ***
%
%\usepackage{array}
% Frank Mittelbach's and David Carlisle's array.sty patches and improves
% the standard LaTeX2e array and tabular environments to provide better
% appearance and additional user controls. As the default LaTeX2e table
% generation code is lacking to the point of almost being broken with
% respect to the quality of the end results, all users are strongly
% advised to use an enhanced (at the very least that provided by array.sty)
% set of table tools. array.sty is already installed on most systems. The
% latest version and documentation can be obtained at:
% http://www.ctan.org/tex-archive/macros/latex/required/tools/

% correct bad hyphenation here
\hyphenation{op-tical net-works semi-conduc-tor}
%\usepackage[Greek,Latin]{ucharclasses}
%\usepackage{xltxtra}

\usepackage{polyglossia} % χρησιμοποιείται για καλύτερη υποστήριξη των Ελληνικών
\usepackage[table]{xcolor} % χρησιμοποιείται για τα χρώματα στους πίνακες
\usepackage{csquotes} % χρησιμοποιείται από το BibLaTeX
\usepackage[backend=biber, style=ieee]{biblatex} %χρησιμοποιείται για τις αναφορές
\usepackage{xpatch} % χρησιμοποιείται για το διόρθωμα ενός bug στο στυλ αναφορών
\usepackage{pgfplots} % χρησιμοποιείται για τα γραφήματα
\usepackage{pgf-pie} % χρησιμοποιείται για πίτες
\usepackage[labelformat=simple]{subcaption} % χρησιμοποιείται για τα υποσχήματα
\usepackage[linesnumbered,ruled,vlined]{algorithm2e} % χρησιμοποιείται για τους αλγόριθμους
\usepackage{booktabs} % χρησιμοποιείται για όμορφους πίνακες
\usepackage{hyperref} % χρησιμοποιείται για αυτόματο linkification

\pgfplotsset{compat=1.10}

\setmainlanguage[numerals=arabic]{greek} % κύρια γλώσσα
\setotherlanguages{english} % δευτερεύουσα γλώσσα

% να μην εμφανίζονται κενές παρενθέσεις όταν δεν υπάρχει ημερομηνία στην αναφορά
\xpatchbibdriver{online}
  {\printtext[parens]{\usebibmacro{date}}}
  {\iffieldundef{year}
    {}
    {\printtext[parens]{\usebibmacro{date}}}}
  {}
  {\typeout{There was an error patching biblatex-ieee (specifically, ieee.bbx's @online driver)}}


% Training subcaption package to comply with
% IEEE standards. We can ignore the warning
% generated by caption.sty which is due to
% the redefinition of \@makecaption
\DeclareCaptionLabelSeparator{periodspace}{.\quad}
\captionsetup{font=footnotesize,labelsep=periodspace,justification=centering,singlelinecheck=false}
\captionsetup[table]{format=plain,labelformat=simple,justification=centering, labelsep=newline, singlelinecheck=false, textfont={sc}}
\captionsetup[sub]{font=footnotesize,singlelinecheck=true}
\renewcommand\thesubfigure{(\alph{subfigure})}

\SetAlgorithmName{Αλγόριθμος}{αλγόριθμος}{Λίστα Αλγορίθμων} % μετάφραση της λέξης «αλγόριθμος»

\addbibresource{bibliography.bib} % όρισμα αρχειού βιβλιογραφίας που θα χρησιμοποιηθεί

% Fonts
\newfontfamily\greekfont[Script=Greek]{Times New Roman} % work-around για bug του polyglossia
\setmainfont[Kerning=On,Mapping=tex-text]{Times New Roman} % roman font
\renewcommand*{\bibfont}{\footnotesize} % αλλαγή του μεγέθους των αναφορών

\renewcommand{\arraystretch}{1.3} % μεγαλύτερο διάστημα ανάμεσα στις γραμμές πινάκων

\begin{document}

% paper title
\title{Εξόντωση σφηκών με χρήση εξελικτικών αλγορίθμων}


% author names and affiliations
\author{\IEEEauthorblockN{Άνι Χατσατριάν}
\IEEEauthorblockA{Τμήμα Μηχανικών Πληροφορικής, ΑΤΕΙΘ\\
Θεσσαλονίκη, Ελλάδα\\
Email: achatsat@it.teithe.gr}
\and
\IEEEauthorblockN{Μιχαήλ Κοσματόπουλος}
\IEEEauthorblockA{Τμήμα Μηχανικών Πληροφορικής, ΑΤΕΙΘ\\
Θεσσαλονίκη, Ελλάδα\\
Email: mkosm@it.teithe.gr}}

% make the title area
\maketitle

% As a general rule, do not put math, special symbols or citations
% in the abstract
\begin{abstract})
The abstract goes here.
\end{abstract}

% no keywords

% For peer review papers, you can put extra information on the cover
% page as needed:
% \ifCLASSOPTIONpeerreview
% \begin{center} \bfseries EDICS Category: 3-BBND \end{center}
% \fi
%
% For peerreview papers, this IEEEtran command inserts a page break and
% creates the second title. It will be ignored for other modes.
\IEEEpeerreviewmaketitle


% Main content
\section{Introduction}
% no \IEEEPARstart
This demo file is intended to serve as a ``starter file''
for IEEE conference papers produced under \LaTeX\ using
IEEEtran.cls version 1.8 and later.
% You must have at least 2 lines in the paragraph with the drop letter
% (should never be an issue)
I wish you the best of success.

\hfill mds

\hfill December 27, 2012

\subsection{Subsection Heading Here}
Subsection text here.


\subsubsection{Subsubsection Heading Here}
Subsubsection text here.
\section{Βιολογικό Υπόβαθρο}

\subsection{Ο Κάρολος και η Θεωρία της Εξέλιξης}
Όπως οι περισσότερες ανακαλύψεις τον ανθρώπων, έτσι και οι εξελικτικοί αλγόριθμοι είναι εμπνευσμένοι από την ίδια τη φύση. Η βάση για τη δημιουργία των εξελικτικών αλγορίθμων είναι η θεωρία της βιολογικής εξέλιξης η οποία προτάθηκε από τον Κάρολο Δαρβίνο στο πολύφημο έργο του "On the Origin of Species by Means of Natural Selection, or The Preservation of Favoured Races in the Struggle for Life".

Ο Δαρβίνος αναλύει την προσαρμοστικότητα των ζώων χρησιμοποιώντας την αρχή της φυσικής επιλογής. Δηλαδή στην ικανότητα των οργανισμών να προσαρμόζονται στο περιβάλλον και να αποκτούν απογόνους. Οι απόγονοι θα έχουν τα ίδια ή παρόμοια χαρακτηριστικά με τους προγόνους τους, έτσι ώστε να μπορέσουν να επιβιώσουν και να συνεχίσουν την αναπαραγωγή του είδους τους. Αυτός είναι ένας από τους κυριότερους μηχανισμούς της φυσικής εξέλιξης.

Η θεωρία του Δαρβίνου βασίζεται σε δεδομένα και στα συμπεράσματα που προκύπτουν από αυτά τα οποία συνόψισε ο βιολόγος Ernst Mayr \cite{Mayr1982} ως εξής:
\begin{itemize}
    \item Κάθε είδος είναι γόνιμο σε τέτοιο βαθμό έτσι ώστε εάν επιβιώσουν όλοι οι απόγονοι τότε ο πληθυσμός τους θα αυξηθεί.
    \item Παρά τις περιοδικές διακυμάνσεις, οι πληθυσμοί εξακολουθούν να έχουν το ίδιο περίπου μέγεθος.
    \item Οι πόροι, όπως τα τρόφιμα και το νερό είναι περιορισμένοι και παραμένουν σταθεροί με την πάροδο του χρόνου. Έτσι προκύπτει ένας αγώνας για την επιβίωση.
    \item Τα άτομα σε έναν πληθυσμό ποικίλλουν σημαντικά το ένα από το άλλο.
    \item Τα άτομα που είναι λιγότερο κατάλληλα για το περιβάλλον έχουν τη μικρότερη πιθανότητα να επιζήσουν και να αναπαραχθούν, ενώ τα άτομα που έχουν όλα τα κατάλληλα χαρακτηριστικά για να επιβιώσουν στο περιβάλλον έχουν την μεγαλύτερη πιθανότητα να επιζήσουν και να αναπαραγάγουν απογόνους οι οποίοι κληρονομούν τα χαρακτηριστικά τους και έτσι δημιουργείται η διαδικασία της φυσικής επιλογής.
    \item Αυτή η σταδιακά αναπτυσσόμενη διαδικασία οδηγεί στην αλλαγή των πληθυσμών για την καλύτερη προσαρμοστικότητα και τελικά με τη συσσώρευση των διακυμάνσεων, καταλήγει στον σχηματισμό νέων ειδών.
\end{itemize}
Επιπρόσθετα, στο έργο του Δαρβίνου αναφέρονται οι διαφορές στα φυσικά και πνευματικά χαρακτηριστικά των οργανισμών \cite{Adamidis} όπως το το ύψος, το βάρος, η ευφυΐα, το χρώμα των ματιών και του τριχώματος. Δηλαδή αναλύονται οι αποκλίσεις ανάμεσα στους φαινότυπους των οργανισμών, οι οποίοι καθορίζουν τον τρόπο ανταπόκρισης και φυσικής ενσάρκωσης των γονέων και των παιδιών τους.

Όλες αυτές οι μεταλλάξεις είναι σημαντικές μόνο εάν δίνουν τη δυνατότητα στον οργανισμό να επιβιώσει σε συγκεκριμένες συνθήκες του περιβάλλοντος. Σε διαφορετική περίπτωση, οι μεταλλάξεις δεν έχουν καμία αξία αφού στο τέλος ο οργανισμός δεν έχει επιβιώνει και καταλήγει να εξαλείφεται.

Στη περίπτωση όμως που οι συνθήκες του περιβάλλοντος είναι ευνοϊκές και οι οργανισμοί συνεχίζουν να αναπαράγονται, ο μόνος περιορισμός που υπάρχει είναι οι διαθεσιμότητα των πόρων. Για αυτό το λόγο, όταν οι πόροι είναι ανεπαρκής, επιβιώνουν μόνο οι οργανισμοί που τους εκμεταλλεύονται πιο αποδοτικά και αποτελεσματικά.

\subsection{Ανάλυση Βιολογικών Χαρακτηριστικών}
Από τη στιγμή που οι εξελικτικοί αλγόριθμοι βασίζονται σε βιολογικά χαρακτηριστικά της φύσης, είναι φανερό ότι για να τους κατανοήσει κανείς χρειάζεται να γνωρίζει κάποια βασικά βιολογικά στοιχεία.

Το σώμα όλων των οργανισμών αποτελείται από εκατομμύρια κύτταρα. Τα περισσότερα κύτταρα περιέχουν μια ολοκληρωμένη σειρά από γονίδια. Διαθέτουν λοιπόν χιλιάδες γονίδια. Τα γονίδια περιέχουν οδηγίες που ελέγχουν την ανάπτυξή και τον τρόπο που δουλεύει το σώμα. Είναι επίσης υπεύθυνα για πολλά φυσικά χαρακτηριστικά, όπως το χρώμα των ματιών, την ομάδα αίματος και το ύψος. Οι πιθανές τιμές κάθε γονιδίου ονομάζονται τιμές χαρακτηριστικών (alleles).

Στους βιολογικούς οργανισμούς \cite{Vlaxavas}, ένα χρωμόσωμα είναι ένα μεγάλο μόριο (ακολουθία) DNA και περιέχει έναν αριθμό γονιδίων. Κάθε γονίδιο έχει μια συγκεκριμένη θέση (locus) μέσα στο χρωμόσωμα. Στο πραγματικό DNA, το αλφάβητο έχει μήκος τέσσερα και αποτελείται από τα γράμματα A, G, T και C που αντιστοιχούν στα τέσσερα διαφορετικά νουκλεοτίδια (βάσεις) που το συνθέτουν (Adenine, Guanine, Thymine και Cytosine). Ένας οργανισμός μπορεί να έχει ένα ή περισσότερα χρωμοσώματα ενώ σε κάποιους οργανισμούς κάθε κύτταρο περιέχει δύο αντίγραφα για κάθε χρωμόσωμα. Για παράδειγμα, ο άνθρωπος έχει 23 ζεύγη χρωμοσωμάτων.

Στην κλασσική προσέγγιση των γενετικών αλγορίθμων, κάθε υποψήφια λύση αναπαριστάται με μία συμβολοσειρά (string) ενός πεπερασμένου αλφαβήτου. Συνήθως χρησιμοποιείται το δυαδικό αλφάβητο, οπότε οι συμβολοσειρές ονομάζονται και δυαδικές συμβολοσειρές (bit-strings). Ωστόσο, υπάρχουν περιπτώσεις γενετικών αλγορίθμων που χρησιμοποιούν πιο σύνθετες μορφές αναπαράστασης. Στα περισσότερα προβλήματα οι λύσεις περιγράφονται με μεταβλητές διαφόρων τύπων δεδομένων, επομένως η διαδικασία της κωδικοποίησης περιλαμβάνει τη μετατροπή των τιμών αυτών των μεταβλητών στις αντίστοιχες δυαδικές.

Κάθε γενετικός αλγόριθμος έχει ένα χρωμόσωμα, δηλαδή μια συμβολοσειρά με πεπερασμένο αριθμό χαρακτήρων. Τα επιμέρους τμήματα της συμβολοσειράς που κωδικοποιούν κάποιο χαρακτηριστικό, δηλαδή κάποια μεταβλητή, αντιπροσωπεύουν τα γονίδια.

Στη γενετική, το σύνολο των παραμέτρων που αναπαρίστανται από ένα συγκεκριμένο γονίδιο που μας ενδιαφέρει ή έναν αριθμό γονιδίων αναφέρεται σαν γονότυπος (genotype). Η συνολική εμφάνιση ενός οργανισμού ή εκδήλωση ενός χαρακτηριστικού ονομάζεται φαινότυπος και εξαρτάται άμεσα από το γονότυπο του.

Κάθε γονότυπος θα μπορούσε να αποτελεί μια πιθανή λύση στο πρόβλημα, η σημασία της οποίας καθορίζεται από το χρήστη. 
\section{Μοντέλα Εξελικτικών Αλγορίθμων}
Όπως αναφέρθηκε πιο πριν, οι εξελικτικοί αλγόριθμοι είναι εμπνευσμένοι από μηχανισμούς της βιολογικής εξέλιξης όπως είναι για παράδειγμα η αναπαραγωγή, η μετάλλαξη, ο ανασυνδυασμός και η φυσική επιλογή. \cite{_evolutionary_2014} Υπάρχουν διάφορα μοντέλα που μπορούν να χρησιμοποιηθούν για την επίλυση ενός προβλήματος με την χρήση εξελικτικών αλγορίθμων. Τα πιο διαδεδομένα από αυτά τα μοντέλα είναι οι Γενετικοί Αλγόριθμοι (Genetic Algorithms), ο Γενετικός Προγραμματισμός (Genetic Programming), οι Εξελικτικές Στρατηγικές (Evolution Strategies) και ο Εξελικτικός Προγραμματισμός (Evolutionary Programming). Οι τεχνικές αυτές δεν διαφέρουν πάρα πολύ μεταξύ τους, παρά μόνο στις λεπτομέρειες της υλοποίησής τους. \cite{Adamidis}

Το μοντέλο που χρησιμοποιείται πιο συχνά για την επίλυση προβλημάτων με την χρήση εξελικτικών αλγόριθμων είναι οι γενετικοί αλγόριθμοι, οι οποίοι αναλύονται παρακάτω.

\subsection{Γενετικοί Αλγόριθμοι}

Οι γενετικοί αλγόριθμοι εφευρέθηκαν την δεκαετία του 1960 από τον John Holland, και αναπτύχθηκαν περισσότερο στο Πανεπιστήμιο του Μίσιγκαν από τον ίδιο, τους φοιτητές του και τους συναδέλφους του. Ο στόχος του Holland ήταν να μελετήσει το φαινόμενο της προσαρμογής όπως αυτό παρατηρείται στη φύση, και όχι να επιλύσει συγκεκριμένα προβλήματα. Ήθελε με αυτόν τον τρόπο να αναπτύξει μεθόδους για την εισαγωγή των μηχανισμών της φυσικής επιλογής σε υπολογιστικά συστήματα. \cite{Melanie1999}

\subsection{Τρόποι Αναπαράστασης}

Ο πιο συνηθισμένος τρόπος αναπαράστασης των χρωμοσωμάτων στους γενετικούς αλγόριθμους είναι η σταθερού-μήκους συμβολοσειρά δυαδικών ψηφίων (bit string). Η χρήση των bit string διευκολύνει την διαδικασία εύρεσης απογόνων.

<ΠΑΡΑΔΕΙΓΜΑ BIT STRING>

Ένα πρόβλημα που παρουσιάζουν τα bit string είναι ο λεγόμενος «γκρεμός Hamming». Για παράδειγμα, ο αριθμός $255$ στο δυαδικό σύστημα γράφεται ως $011111111$ και ο αριθμός $256$ ως $100000000$. Για την αλλαγή από τον έναν αριθμό στον άλλον πρέπει να αλλάξουν $9$ bits, που σημαίνει ότι είναι πολύ πιθανό να πρέπει να βρεθούν πολλοί απόγονοι μέχρι το χρωμόσωμα να φτάσει σε μια πιο βέλτιστη λύση. Αυτό το πρόβλημα μπορεί να αντιμετωπιστεί κωδικοποιώντας τους αριθμούς με τον κώδικα Gray. Ό κώδικας Gray είναι ένα δυαδικό σύστημα όπου δύο διαδοχικοί αριθμοί διαφέρουν μόνο κατά ένα bit. \cite{Lehre}

\begin{table}[!t]
    \renewcommand{\arraystretch}{1.3}
    \caption{Κωδικοποίηση Gray 3 bit}
    \label{table_gray_code}
    \centering
    \begin{tabular}{c|c|c}
        \hline
        \bfseries Δεκαδικό & \bfseries Δυαδικό & \bfseries Κώδικας Gray\\
        \hline\hline
        0 & 000 & 000\\
        1 & 001 & 001\\
        2 & 010 & 011\\
        3 & 011 & 010\\
        4 & 100 & 110\\
        5 & 101 & 111\\
        6 & 110 & 101\\
        7 & 111 & 100\\
        \hline
    \end{tabular}
\end{table}

Άλλοι τρόποι αναπαράστασης που μπορούν να χρησιμοποιηθούν ανάλογα με το πρόβλημα είναι οι μεταθέσεις και η δεντρική δομή. Οι μεταθέσεις χρησιμοποιούνται συνήθως σε προβλήματα ταξινόμησης, όπως για παράδειγμα το γνωστό πρόβλημα του περιπλανώμενου πωλητή (το κάθε χρωμόσωμα δίνει την σειρά που ο πωλητής θα επισκεφθεί τις πόλεις). \cite{Obitko}

\subsection{Δημιουργία Πληθυσμού}

Στη φάση της δημιουργίας του αρχικού πληθυσμού, συνήθως δημιουργούνται Ν τυχαία σταθερού-μήκους bit strings, όπου Ν είναι το μέγεθος του πληθυσμού που έχει οριστεί. Υπάρχουν έρευνες που αποδεικνύουν ότι όσο μεγαλύτερο το μέγεθος του πληθυσμού, τόσο μεγαλύτερη η ακρίβεια του γενετικού αλγόριθμου, αλλά και τόσες περισσότερες γενιές χρειάζονται για να αυξηθεί η σύγκλιση. \cite{Gotshall2008}

\subsection{Επιλογή Γονέων}

Αφού δημιουργηθεί ο αρχικός πληθυσμός, θα πρέπει να επιλεχθούν τα χρωμοσώματα που θα γίνουν γονείς και θα παράγουν τα χρωμοσώματα-παιδιά που θα χρησιμοποιηθούν για τον επόμενο κύκλο του γενετικού αλγόριθμου. Είναι επιθυμητό τα χρωμοσώματα που δίνουν καλύτερη λύση στο πρόβλημα να έχουν περισσότερες πιθανότητες να επιλεχθούν. Για τον έλεγχο της ποιότητας ενός χρωμοσώματος, χρησιμοποιείται η συνάρτηση καταλληλότητας (fitness function). Η συνάρτηση καταλληλότητας δέχεται σαν είσοδο ένα χρωμόσωμα, και επιστρέφει έναν αριθμό στο διάστημα [0, 1], όπου το 0 υποδηλώνει ότι το χρωμόσωμα δεν αποτελεί καλή λύση του προβλήματος, ενώ το 1 υποδηλώνει ότι το χρωμόσωμα είναι τέλειο.

Υπάρχουν διάφορες τεχνικές για την επιλογή των γονέων. Μία από αυτές τις τεχνικές είναι η επιλογή ρουλέτας (roulette wheel selection) που αναφέρθηκε παραπάνω, όπου τα χρωμοσώματα καταλαμβάνουν χώρο στη «ρουλέτα» ανάλογα με την ποιότητα τους (οπότε υπάρχει μεγαλύτερη πιθανότητα να «πέσει η μπίλια» στα καλύτερα χρωμοσώματα). Μια άλλη τεχνική ονομάζεται επιλογή τουρνουά (tournament selection), και λειτουργεί επιλέγοντας ένα τυχαίο υποσύνολο k χρωμοσωμάτων από τον αρχικό πληθυσμό, και επιλέγοντας το καλύτερο χρωμόσωμα από αυτό το υποσύνολο.

\subsection{Αναπαραγωγή των Γονέων}

Μετά την χρήση μιας από τις παραπάνω μεθόδους για την επιλογή των χρωμοσωμάτων που θα χρησιμοποιηθούν ως γονείς στην αναπαραγωγή, πρέπει να εφαρμοστεί ένας ή περισσότεροι τελεστές για την δημιουργία των χρωμοσωμάτων-παιδιών. Μερικοί από τους τελεστές που χρησιμοποιούνται συχνότερα είναι η διασταύρωση (crossover) και η μετάλλαξη.

\subsubsection{Διασταύρωση ενός σημείου}

Στη διασταύρωση ενός σημείου (one-point crossover) επιλέγεται τυχαία ένα σημείο στο χρωμόσωμα των γονέων, και τα δεδομένα μετά από αυτό το σημείο ανταλλάσσονται μεταξύ των δύο γονέων. Για παράδειγμα, έστω ότι τα χρωμοσώματα $0110100$ και $1100110$ είναι οι δύο γονείς, και το σημείο διασταύρωσης είναι το $4$. Τότε τα καινούργια χρωμοσώματα θα είναι τα $0110110$ και $1100100$.

\begin{figure}[!t]
    \centering
    \def\svgwidth{2.5in}
    \input{./figures/OnePointCrossover.pdf_tex}
    \caption{Διασταύρωση ενός σημείου \cite{opc_fig}}
    \label{fig_opc}
\end{figure}

\subsubsection{Διασταύρωση δύο σημείων}

Στη διασταύρωση δύο σημείων (two-point crossover), επιλέγονται τυχαία δύο σημεία στο χρωμόσωμα των γονέων, και τα δεδομένα που βρίσκονται ανάμεσα στα δύο σημεία ανταλλάσσονται μεταξύ των δύο γονέων. Για παράδειγμα, έστω πάλι τα χρωμοσώματα $0110100$ και $1100110$, και τα δύο τυχαία σημεία είναι το 2 και το 5. Τότε, τα παιδιά που θα παραχθούν θα είναι τα $0100100$ και $1110110$.

\begin{figure}[!t]
    \centering
    \def\svgwidth{2.5in}
    \input{./figures/TwoPointCrossover.pdf_tex}
    \caption{Διασταύρωση δύο σημείων \cite{tpc_fig}}
    \label{fig_tpc}
\end{figure}

\subsubsection{Μετάλλαξη}

Στη μετάλλαξη (mutation) επιλέγεται τυχαία ένα bit από το χρωμόσωμα, και αυτό αντιστρέφεται (δηλαδή από $0$ γίνεται $1$, και από $1$ γίνεται $0$). Αυτός ο τελεστής πρέπει να έχει μικρή πιθανότητα να εφαρμοστεί (κοντά στο $1\%$), αλλιώς υπάρχει κίνδυνος η αναζήτηση να μετατραπεί σε τυχαία αναζήτηση. \cite{zotero-PNQJ3TKI}

<ΣΧΗΜΑ mutation (?)>
\section{Επίλυση}

\subsection{Ανάλυση Βιβλιοθήκης JGAP}

Η υλοποίηση των εξελικτικών αλγορίθμων απαιτεί πολύ χρόνο και κόπο. Για αυτό το λόγο έχουν αναπτυχθεί διάφορες βιβλιοθήκες οι οποίες παρέχουν όλους τους βασικούς  μηχανισμούς για την υλοποίηση ενός εξελικτικού αλγορίθμου.

Μια τέτοια βιβλιοθήκη που επιτρέπει την ανάπτυξη εξελικτικών αλγορίθμων είναι η JGAP. Η βιβλιοθήκη αυτή έχει αναπτυχθεί στη γλώσσα Java και υποστηρίζει μεγάλη προσαρμοστικότητα χωρίς όμως αυτό να σημαίνει ότι είναι δύσκολη στη χρήση. Στην πραγματικότητα συμβαίνει το ακριβώς αντίθετο.

Παρέχει όλες τις απαραίτητες κλάσεις και μεθόδους οι οποίες μπορούν να χρησιμοποιηθούν για την εφαρμογή των εξελικτικών αρχών στις λύσεις των προβλημάτων. \cite{Meffert}

Για να αναπτυχθεί ένας γενετικός αλγόριθμος με τη χρήση της βιβλιοθήκης JGAP, θα πρέπει να πραγματοποιηθούν τα εξής βήματα: \cite{zotero-R7UV9N25}
\begin{enumerate}
  \item Σχεδιασμός χρωμοσώματος
  \item Υλοποίηση συνάρτησης καταλληλότητας
  \item Ορισμός των παραμέτρων
  \item Δημιουργία ενός πληθυσμού από υποψήφιες λύσεις
  \item Εξέλιξη του πληθυσμού
\end{enumerate}

Αξίζει λοιπόν να γίνει μια αναφορά για κάθε βήμα ξεχωριστά.

\subsubsection{Σχεδιασμός Χρωμοσώματος}

Όπως αναφέρθηκε, το χρωμόσωμα αναπαριστά μια υποψήφια λύση ενός προβλήματος και αποτελείται από πολλαπλά γονίδια. Τα γονίδια στην βιβλιοθήκη JGAP αναπαριστούν διακριτά χαρακτηριστικά της λύσης. Για αυτό το λόγο το πρώτο βήμα αφορά τον ορισμό του χρωμοσώματος. Δηλαδή στην ουσία γίνεται ο ορισμός των γονιδίων (genes).

Αυτό γίνεται με την υλοποίηση της διεπαφής Gene η οποία αντιπροσωπεύει ένα γονίδιο χρωμοσώματος. Τέτοιες κλάσεις είναι η IntegerGene, DoubleGene και άλλες.

Για την δημιουργία του χρωμοσώματος, τα γονίδια τοποθετούνται σε ένα πίνακα και στη συνέχεια δημιουργείται ένα αντικείμενο τύπου Chromosome το οποίο δέχεται σαν όρισμα το πίνακα με τα γονίδια.

\subsubsection{Υλοποίηση Συνάρτησης Καταλληλότητας}

Για την αξιολόγηση των χρωμοσωμάτων είναι αναγκαία η χρήση μιας συνάρτησης καταλληλότητας. Στη βιβλιοθήκη JGAP, αυτό επιτυχγάνεται με την δημιουργίας μιας κλάσης που επεκτείνει την ήδη υπάρχουσα αφηρημένη κλάση FitnessFunction.

Σε αυτήν θα πρέπει να υλοποιηθεί η μέθοδος evaluate, η οποία δέχεται σαν όρισμα ένα χρωμόσωμα και αναλόγως επιστρέφει την τιμή της καταλληλότητας του. Είναι προκαθορισμένο για τις τιμές που επιστρέφει η μέθοδος evaluate να μην είναι αρνητικές, άρα η τιμή της καταλληλότητας δεν θα πρέπει να είναι αρνητικός αριθμός.

\subsubsection{Ορισμός των Παραμέτρων}

Η βιβλιοθήκη JGAP προσφέρει για κάθε εξελικτικό αλγόριθμο μια πληθώρα από παραμετροποιήσεις και επιλογές. Οι παραμετροποιήσεις γίνονται με τη χρήση του αντικειμένου Configuration, το οποίο πρέπει να δημιουργείται πριν την εκτέλεση του αλγορίθμου.

Μερικές από τις σημαντικότερες αλλαγές που μπορούν να γίνουν μέσω του αντικειμένου Configuration, αφορούν τους γενετικούς τελεστές, τον τρόπο που θα δημιουργούνται οι τυχαίοι αριθμοί και τον τρόπο με τον οποίο θα υλοποιείται η φυσική επιλογή.

Για την διευκόλυνση των χρηστών, υπάρχει μια κλάση που ονομάζεται DefaultConfiguration και περιέχει τις προκαθορισμένες παραμέτρους.Ειδικότερα, για την επιλογή χρησιμοποιείται η κλάση BestChromosomesSelector και οι γενετικοί τελεστές της διασταύρωσης και μετάλλαξης έχουν τις τιμές 0.35 και 12 αντιστοίχως.

\subsubsection{Δημιουργία ενός Πληθυσμού από Υποψήφιες Λύσεις}

Αφού ολοκληρωθούν τα παραπάνω βήματα φτάνει η στιγμή που πρέπει να δημιουργηθεί ο αρχικός πληθυσμός των χρωμοσωμάτων, οποίος ονομάζεται γονότυπος. Για αυτό το λόγο η κλάση που χρησιμοποιείται για τη δημιουργία του πληθυσμού ονομάζεται Genotype.

Αυτή η κλάση περιέχει τη μέθοδο randomInitialGenotype η οποία δέχεται το αντικείμενο Configuration, το οποίο περιέχει όλες τις παραμέτρους μας και με βάση αυτό, δημιουργεί τον σωστό αριθμό χρωμοσωμάτων. Τις περισσότερες φορές είναι αναγκαία η δημιουργία ενός αρχικού πληθυσμού.

\subsubsection{Εξέλιξη του Πληθυσμού}

Το τελικό βήμα για την ολοκλήρωση του αλγορίθμου είναι η εξέλιξη του πληθυσμού. Αυτή πραγματοποιείται με το κάλεσμα της μεθόδου evolve η οποία βρίσκεται στη κλάση Genotype. Η εξέλιξη, δηλαδή το κάλεσμα της evolve μπορεί να πραγματοποιηθεί όσες φορές επιθυμεί ο χρήστης.

Η ανάκτηση του καλύτερου αποτελέσματος γίνεται με τη μέθοδο getFittestChromosome. Αυτή επιστρέφει το χρωμόσωμα με τη μεγαλύτερη καταλληλότητα, δηλαδή το αντικείμενο Chromosome που εμφάνισε τη μεγαλύτερη τιμή στη μέθοδο evaluate.

\subsection{Σενάριο Προβλήματος}

Μόλις αγοράσατε ένα σπίτι και ανακαλύπτετε ότι η σοφίτα του είναι γεμάτη από σφηκοφωλιές. Πριν μετακομίσετε στο νέο σας σπίτι αποφασίζετε να εξοντώσετε τις σφήκες. Επισκέπτεστε το κατάστημα της περιοχής σας το οποίο διαθέτει εντομοκτόνα αλλά βρίσκεται μόνο τρία (3) δοχεία τύπου «εντομο-βόμβας» τα οποία έχουν συγκεκριμένη
ακτίνα δράσης και πρέπει να τοποθετηθούν πολύ κοντά στη φωλιά για να εξοντώσουν τις σφήκες που βρίσκονται μέσα. Δυστυχώς τα 3 δοχεία δεν είναι αρκετά να εξοντώσουν όλες τις σφήκες της σοφίτας.

Ευτυχώς η τύχη σας βοηθάει και βρίσκετε:

\begin{itemize}
  \item  έναν χάρτη (σχήμα \ref{fig_waspNestsMap}) που άφησε ο προηγούμενος ιδιοκτήτης και ο οποίος περιγράφει την θέση που βρίσκονται οι φωλιές όπως επίσης και τον αριθμό σφηκών που διαθέτει κάθε φωλιά (χρησιμοποιώντας ένα πίνακα 100x100),
  \item έναν τύπο (εξίσωση \ref{eq_bugBomb}) πάνω στο δοχείο «εντομο-βόμβας» ο οποίος δίνει την σχέση απόστασης από την φωλιά και του ποσοστού των σφηκών οι οποίες εξοντώνονται.
      \begin{equation}\label{eq_bugBomb}
        K = n* \frac{dmax}{20*d+0.00001}
      \end{equation}
      όπου\\
      $K$: Πλήθος σφηκών που θα σκοτωθούν σε μία φωλιά\\
      $n$: Πλήθος υπαρχόντων σφηκών σε αυτή τη φωλιά\\
      $d$: Απόσταση βόμβας από αυτή τη φωλιά\\
      $dmax$: Η μέγιστη απόσταση μεταξύ δύο φωλιών\\
\end{itemize}

\begin{figure}[!t]
    \centering
    \begin{tikzpicture}
    	\begin{axis}[
            xlabel={x},
            ylabel={y},
            width = 0.8\columnwidth,
            %height = 0.8\columnwidth,
            colorbar,
            legend cell align = left,
            legend pos = north west,
            %colormap = {whiteblack}{gray(0cm) = (1); gray(1cm) = (0)},
            colorbar style = {title=Αριθμός Σφηκών, /tikz/.cd},
            font = \footnotesize]

            \addplot +[scatter, only marks, point meta=explicit] table [meta=wasps, col sep=comma] {./figures/waspNests.csv};
            \addlegendentry{Σφηκοφωλιά}
        \end{axis}
    \end{tikzpicture}
    \caption{Χάρτης σφηκοφωλιών}
    \label{fig_waspNestsMap}
\end{figure}

Η απόσταση μεταξύ δύο θέσεων του χάρτη υπολογίζεται από την εξίσωση \ref{eq_distance}
\begin{equation}\label{eq_distance}
    d = \sqrt{(x_{1}-x_{2})^2+(y_{1}-y_{2})^2}
\end{equation}

Στόχος είναι να βρεθεί η καλύτερη δυνατή τοποθέτηση των δοχείων έτσι ώστε να εξοντωθεί ο μεγαλύτερος αριθμός σφηκών.

\subsection{Ανάλυση Προβλήματος}

Το πρόβλημα που τίθεται προς λύση από τη μία πλευρά μπορεί να θεωρηθεί πρόβλημα μεγιστοποίησης, εάν θεωρήσει κανείς ότι το ζητούμενο του προβλήματος είναι η εξαγωγή του αριθμού των εξοντωμένων σφηκών. Με αυτή τη σκοπιά στη συνάρτηση καταλληλότητας επιλέγονται τα χρωμοσώματα που οδηγούν στη μεγαλύτερη τιμή, ενώ απορρίπτονται αυτά που εξάγουν μικρό αριθμό εξοντωμένων σφηκών.

Μπορεί όμως να θεωρηθεί εξίσου ως ένα πρόβλημα ελαχιστοποίησης, με το σκεπτικό ότι το ζητούμενο είναι η μείωση του συνολικού αριθμού σφηκών οι οποίες βρίσκονται στην σοφίτα. Έτσι, στη συνάρτηση καταλληλότητας επιλέγονται τα χρωμοσώματα που οδηγούν στο μικρότερο αριθμό σφηκών που επέζησαν μετά τις εκρήξεις και απορρίπτονται αυτά που άφησαν τον μεγαλύτερο αριθμό σφηκών ζωντανό.

Στην παρούσα εργασία προτιμήθηκε η πρώτη επιλογή, χωρίς αυτό να σημαίνει ότι είναι καλύτερη από την άλλη. Ο λόγος της επιλογής αυτής είναι καθαρά για λόγους ευκολίας ως προς τη χρήση της βιβλιοθήκης JGAP.

\subsection{Περιγραφή Λύσης}

Αρχικά, δημιουργήθηκαν οι κλάσεις WaspNest και Map\-Controller.

Η κλάση WaspNest κρατάει τις πληροφορίες της κάθε σφηκοφωλιάς (τοποθεσία και αριθμός σφηκών), ενώ επίσης παρέχει την μέθοδο killWasps(killedWasps); η οποία αφαιρεί από τον αριθμό των σφηκών της συγκεκριμένης φωλιάς τον αριθμό killedWasps. Σε περίπτωση που η διαφορά είναι $\leq 0$, τότε θεωρείται ότι η φωλιά δεν έχει καμία ζωντανή σφήκα.

Η κλάση MapController κρατάει σε έναν πίνακα όλες τις σφηκοφωλιές. Ενδιαφέρον παρουσιάζει η μέθοδος getBomb\-TotalKills(bomb); η οποία υπολογίζει πόσες σφήκες θα πεθάνουν συνολικά αν σκάσει η βόμβα bomb, και καλεί την μέθοδο killWasps(); της κάθε σφηκοφωλιάς ώστε να ενημερωθεί το πλήθος των ζωντανών σφηκών.

Επειδή το αποτέλεσμα του τύπου \ref{eq_bugBomb} εξαρτάται και από τον αριθμό των σφηκών που έχει η φωλιά, έχει σημασία η σειρά με την οποία θα σκάσουν οι βόμβες. Αυτός είναι και ο λόγος που είναι απαραίτητη η κλήση της μεθόδου killWasps(); όταν σκάει μια βόμβα. Ως εκ τούτου, είναι απαραίτητο να υπάρχει ένας τρόπος ο οποίος θα επαναφέρει τον αριθμό των σφηκών που περιέχει κάθε σφηκοφωλιά στην αρχική του κατάσταση. Για αυτόν τον λόγο δημιουργήθηκε η μέθοδος saveMap(); η οποία αποθηκεύει την κατάσταση του χάρτη και καλείται αμέσως μετά την αρχικοποίηση του χάρτη, και η μέθοδος restoreMap(); που επαναφέρει τον χάρτη σε προηγούμενή του κατάσταση και καλείται μετά τον υπολογισμό του συνολικού αριθμού σφηκών που εξοντώνουν οι τρεις βόμβες. Αυτό γίνεται έτσι ώστε η επόμενη πιθανή λύση του προβλήματος που θα αξιολογηθεί να μην επηρεαστεί από την προηγούμενη.

Όπως αναφέρθηκε και πριν, το πρώτο βήμα για την ανάπτυξη ενός γενετικού αλγορίθμου σε JGAP είναι ο σχεδιασμός του χρωμοσώματος. Στο συγκεκριμένο πρόβλημα, κάθε χρωμόσωμα αποτελείται από έξι γονίδια \textemdash{} δύο γονίδια για τις συντεταγμένες της κάθε βόμβας (σχήμα \ref{fig_chromosomeStructure}).

\begin{figure}[!t]
    \centering
    \begin{tabular}{|c|c|c|c|c|c|}
        \hline
        $x_1$ & $y_1$ & $x_2$ & $y_2$ & $x_3$ & $y_3$\\
        \hline
    \end{tabular}
    \caption{Δομή χρωμοσώματος}
    \label{fig_chromosomeStructure}
\end{figure}

Στη συνέχεια, πρέπει να οριστεί η συνάρτηση καταλληλότητας με την οποία θα αξιολογούνται τα χρωμοσώματα. Αυτό γίνεται στην κλάση στην μέθοδο evaluate της κλάσης WaspFitnessFunction. η οποία καλεί την μέθοδο getBombTotalKills(); για κάθε βόμβα και τέλος, επαναφέρει τον χάρτη στην αρχική του κατάσταση με την μέθοδο restoreMap();.

Οι σημαντικότερες παράμετροι που μπορούν να ρυθμιστούν στο JGAP είναι η μέθοδος επιλογής χρωμοσώματος, η πιθανότητα που μπορεί να συμβεί ανασυνδυασμός ή μετάλλαξη και αν θα χρησιμοποιηθεί ελιτισμός ή όχι.

Το JGAP προσφέρει αρκετούς τρόπους για την επιλογή των χρωμοσωμάτων. Κυριότερες από αυτές είναι ο BestChromosomesSelector που παίρνει τα καλύτερα $n$ χρωμοσώματα στην επόμενη γενιά \cite{BCS}, ο WeightedRouletteSelector που λειτουργεί όπως η επιλογή ρουλέτας που αναφέρθηκε προηγουμένως \cite{WRS}, και ο TournamentSelector που λειτουργεί με παρόμοιο τρόπο με την επιλογή τουρνουά \cite{TS}. Για να διαπιστωθεί ποιος από τους τρεις τρόπους επιλογής δίνει καλύτερα αποτελέσματα, πραγματοποιήθηκαν τρία διαφορετικά πειράματα, όπου στο κάθε πείραμα έμεινε σταθερό το μέγεθος του πληθυσμού και ο μέγιστος αριθμός γενεών (100 και 1000 αντίστοιχα), και άλλαζε ο τρόπος επιλογής του χρωμοσώματος. Το κάθε πείραμα έτρεξε 100 φορές, και μετά πάρθηκε ο μέσος όρος της καταλληλότητας του καλύτερου χρωμοσώματος ανά γενιά, για κάθε τρόπο επιλογής (σχήμα \ref{fig_selectorFitness}). Όπως φαίνεται, ο BestChromosomesSelector και TournamentSelector είχαν σχεδόν ίδιες επιδόσεις, ενώ ο WeightedRouletteSelector άργησε να συγκλίνει σε μια καλύτερη λύση.

\begin{figure}[!t]
    \centering
    \begin{tikzpicture}
        \begin{axis}[
            xlabel={Αριθμός γενιάς},
            ylabel={Καταλληλότητα},
            %ymin = 2300,
            legend cell align = left,
            legend pos = south east,
            font = \footnotesize]

            \addplot +[mark=none] table [x=gen, y=bestchrom, col sep=comma] {./figures/selectors.csv};
            \addlegendentry{BestChromosomesSelector}
            \addplot +[mark=none] table [x=gen, y=roulette, col sep=comma] {./figures/selectors.csv};
            \addlegendentry{WeightedRouletteSelector}
            \addplot +[mark=none] table [x=gen, y=tournament, col sep=comma] {./figures/selectors.csv};
            \addlegendentry{TournamentSelector}
        \end{axis}
    \end{tikzpicture}
    \caption{Μέσος όρος καλύτερου fitness value ανά γενιά}
    \label{fig_selectorFitness}
\end{figure}

Με τον ίδιο τρόπο δοκιμάστηκε το πως επηρεάζει το μέγεθος του πληθυσμού την τιμή της καλύτερης καταλληλότητας ανά γενιά (σχήμα \ref{fig_avgFitness}). Παρατηρείται ότι όσο πιο μεγάλος ο πληθυσμός, τόσο καλύτερη λύση βρίσκεται.

\begin{figure}[!t]
    \centering
    \begin{tikzpicture}
        \begin{axis}[
            xlabel={Αριθμός γενιάς},
            ylabel={Καταλληλότητα},
            %ymin = 2300,
            legend cell align = left,
            legend pos = south east,
            font = \footnotesize]

            \addplot +[mark=none] table [x=gen, y=pop20, col sep=comma] {./figures/averageFitness.csv};
            \addlegendentry{πληθ. = 20}
            \addplot +[mark=none] table [x=gen, y=pop50, col sep=comma] {./figures/averageFitness.csv};
            \addlegendentry{πληθ. = 50}
            \addplot +[mark=none] table [x=gen, y=pop100, col sep=comma] {./figures/averageFitness.csv};
            \addlegendentry{πληθ. = 100}
            \addplot +[mark=none] table [x=gen, y=pop1000, col sep=comma] {./figures/averageFitness.csv};
            \addlegendentry{πληθ. = 1000}
        \end{axis}
    \end{tikzpicture}
    \caption{Μέσος όρος καλύτερου fitness value ανά γενιά}
    \label{fig_avgFitness}
\end{figure}

Τέλος, δοκιμάστηκε η αλλαγή της πιθανότητας του να συμβεί ανασυνδυασμός σε διάφορες ρυθμίσεις (σχήμα \ref{fig_crossover}).

\begin{figure}[!t]
    \centering
    \begin{tikzpicture}
        \begin{axis}[
            xlabel={Πιθανότητα ανασυνδυασμού},
            ylabel={Καταλληλότητα},
            ymin = 2300,
            legend cell align = left,
            legend pos = south east,
            font = \footnotesize]

            \addplot +[mark=none] table [x=a, y=b, col sep=comma] {./figures/crossoverRateFitness.csv};
            \addlegendentry{πληθ. = 50, γεν. = 5}
            \addplot +[mark=none] table [x=a, y=c, col sep=comma] {./figures/crossoverRateFitness.csv};
            \addlegendentry{πληθ. = 50, γεν. = 10}
            \addplot +[mark=none] table [x=a, y=d, col sep=comma] {./figures/crossoverRateFitness.csv};
            \addlegendentry{πληθ. = 20, γεν. = 5}
            \addplot +[mark=none] table [x=a, y=e, col sep=comma] {./figures/crossoverRateFitness.csv};
            \addlegendentry{πληθ. = 20, γεν. = 10}
        \end{axis}
    \end{tikzpicture}
    \caption{Μέσος όρος καλύτερου fitness value καθώς μεγαλώνει το ποσοστό ανασυνδυασμού}
    \label{fig_crossover}
\end{figure}

Αξίζει να σημειωθεί εδώ ότι η βέλτιστη λύση είναι να σκοτωθούν 3054 σφήκες\footnote{Βρέθηκε με μέθοδο εξαντλητικών δοκιμών. Ο κώδικας μπορεί να βρεθεί εδώ: https://github.com/sortingbubbles/wasp-control/tree/bruteforce}, αλλά καμία από τις λύσεις που βρέθηκαν δεν ήταν η βέλτιστη λύση (πολλές φορές όμως ήταν πολύ κοντά στην βέλτιστη). 

\section{Conclusion}
The συμπέρασμα goes here.

% conference papers do not normally have an appendix

% use section* for acknowledgement
\section*{Acknowledgment}

The authors would like to thank...

% references section

\begingroup
\raggedright
\printbibliography
\endgroup
% that's all folks
\end{document}


