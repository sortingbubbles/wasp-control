\documentclass[conference]{IEEEtran}

% *** GRAPHICS RELATED PACKAGES ***
%
\ifCLASSINFOpdf
  % \usepackage[pdftex]{graphicx}
  % declare the path(s) where your graphic files are
  % \graphicspath{{../pdf/}{../jpeg/}}
  % and their extensions so you won't have to specify these with
  % every instance of \includegraphics
  % \DeclareGraphicsExtensions{.pdf,.jpeg,.png}
\else
  % or other class option (dvipsone, dvipdf, if not using dvips). graphicx
  % will default to the driver specified in the system graphics.cfg if no
  % driver is specified.
  % \usepackage[dvips]{graphicx}
  % declare the path(s) where your graphic files are
  % \graphicspath{{../eps/}}
  % and their extensions so you won't have to specify these with
  % every instance of \includegraphics
  % \DeclareGraphicsExtensions{.eps}
\fi
% graphicx was written by David Carlisle and Sebastian Rahtz. It is
% required if you want graphics, photos, etc. graphicx.sty is already
% installed on most LaTeX systems. The latest version and documentation
% can be obtained at:
% http://www.ctan.org/tex-archive/macros/latex/required/graphics/
% Another good source of documentation is "Using Imported Graphics in
% LaTeX2e" by Keith Reckdahl which can be found at:
% http://www.ctan.org/tex-archive/info/epslatex/
%
% latex, and pdflatex in dvi mode, support graphics in encapsulated
% postscript (.eps) format. pdflatex in pdf mode supports graphics
% in .pdf, .jpeg, .png and .mps (metapost) formats. Users should ensure
% that all non-photo figures use a vector format (.eps, .pdf, .mps) and
% not a bitmapped formats (.jpeg, .png). IEEE frowns on bitmapped formats
% which can result in "jaggedy"/blurry rendering of lines and letters as
% well as large increases in file sizes.
%
% You can find documentation about the pdfTeX application at:
% http://www.tug.org/applications/pdftex

% *** SPECIALIZED LIST PACKAGES ***
%
%\usepackage{algorithmic}
% algorithmic.sty was written by Peter Williams and Rogerio Brito.
% This package provides an algorithmic environment fo describing algorithms.
% You can use the algorithmic environment in-text or within a figure
% environment to provide for a floating algorithm. Do NOT use the algorithm
% floating environment provided by algorithm.sty (by the same authors) or
% algorithm2e.sty (by Christophe Fiorio) as IEEE does not use dedicated
% algorithm float types and packages that provide these will not provide
% correct IEEE style captions. The latest version and documentation of
% algorithmic.sty can be obtained at:
% http://www.ctan.org/tex-archive/macros/latex/contrib/algorithms/
% There is also a support site at:
% http://algorithms.berlios.de/index.html
% Also of interest may be the (relatively newer and more customizable)
% algorithmicx.sty package by Szasz Janos:
% http://www.ctan.org/tex-archive/macros/latex/contrib/algorithmicx/

% *** ALIGNMENT PACKAGES ***
%
%\usepackage{array}
% Frank Mittelbach's and David Carlisle's array.sty patches and improves
% the standard LaTeX2e array and tabular environments to provide better
% appearance and additional user controls. As the default LaTeX2e table
% generation code is lacking to the point of almost being broken with
% respect to the quality of the end results, all users are strongly
% advised to use an enhanced (at the very least that provided by array.sty)
% set of table tools. array.sty is already installed on most systems. The
% latest version and documentation can be obtained at:
% http://www.ctan.org/tex-archive/macros/latex/required/tools/

% correct bad hyphenation here
\hyphenation{op-tical net-works semi-conduc-tor}
%\usepackage[Greek,Latin]{ucharclasses}
%\usepackage{xltxtra}

\usepackage{polyglossia} % χρησιμοποιείται για καλύτερη υποστήριξη των Ελληνικών
\usepackage[table]{xcolor} % χρησιμοποιείται για τα χρώματα στους πίνακες
\usepackage{csquotes} % χρησιμοποιείται από το BibLaTeX
\usepackage[backend=biber, style=ieee]{biblatex} %χρησιμοποιείται για τις αναφορές
\usepackage{xpatch} % χρησιμοποιείται για το διόρθωμα ενός bug στο στυλ αναφορών
\usepackage{pgfplots} % χρησιμοποιείται για τα γραφήματα
\usepackage{pgf-pie} % χρησιμοποιείται για πίτες
\usepackage[labelformat=simple]{subcaption} % χρησιμοποιείται για τα υποσχήματα
\usepackage[linesnumbered,ruled,vlined]{algorithm2e} % χρησιμοποιείται για τους αλγόριθμους
\usepackage{booktabs} % χρησιμοποιείται για όμορφους πίνακες
\usepackage{hyperref} % χρησιμοποιείται για αυτόματο linkification

\pgfplotsset{compat=1.10}

\setmainlanguage[numerals=arabic]{greek} % κύρια γλώσσα
\setotherlanguages{english} % δευτερεύουσα γλώσσα

% να μην εμφανίζονται κενές παρενθέσεις όταν δεν υπάρχει ημερομηνία στην αναφορά
\xpatchbibdriver{online}
  {\printtext[parens]{\usebibmacro{date}}}
  {\iffieldundef{year}
    {}
    {\printtext[parens]{\usebibmacro{date}}}}
  {}
  {\typeout{There was an error patching biblatex-ieee (specifically, ieee.bbx's @online driver)}}


% Training subcaption package to comply with
% IEEE standards. We can ignore the warning
% generated by caption.sty which is due to
% the redefinition of \@makecaption
\DeclareCaptionLabelSeparator{periodspace}{.\quad}
\captionsetup{font=footnotesize,labelsep=periodspace,justification=centering,singlelinecheck=false}
\captionsetup[table]{format=plain,labelformat=simple,justification=centering, labelsep=newline, singlelinecheck=false, textfont={sc}}
\captionsetup[sub]{font=footnotesize,singlelinecheck=true}
\renewcommand\thesubfigure{(\alph{subfigure})}

\SetAlgorithmName{Αλγόριθμος}{αλγόριθμος}{Λίστα Αλγορίθμων} % μετάφραση της λέξης «αλγόριθμος»

\addbibresource{bibliography.bib} % όρισμα αρχειού βιβλιογραφίας που θα χρησιμοποιηθεί

% Fonts
\newfontfamily\greekfont[Script=Greek]{Times New Roman} % work-around για bug του polyglossia
\setmainfont[Kerning=On,Mapping=tex-text]{Times New Roman} % roman font
\renewcommand*{\bibfont}{\footnotesize} % αλλαγή του μεγέθους των αναφορών

\renewcommand{\arraystretch}{1.3} % μεγαλύτερο διάστημα ανάμεσα στις γραμμές πινάκων

\begin{document}

% paper title
\title{Εξόντωση σφηκών με χρήση εξελικτικών αλγορίθμων}


% author names and affiliations
\author{\IEEEauthorblockN{Άνι Χατσατριάν}
\IEEEauthorblockA{Τμήμα Μηχανικών Πληροφορικής, ΑΤΕΙΘ\\
Θεσσαλονίκη, Ελλάδα\\
Email: achatsat@it.teithe.gr}
\and
\IEEEauthorblockN{Μιχαήλ Κοσματόπουλος}
\IEEEauthorblockA{Τμήμα Μηχανικών Πληροφορικής, ΑΤΕΙΘ\\
Θεσσαλονίκη, Ελλάδα\\
Email: mkosm@it.teithe.gr}}

% make the title area
\maketitle

% As a general rule, do not put math, special symbols or citations
% in the abstract
\begin{abstract})
The abstract goes here.
\end{abstract}

% no keywords

% For peer review papers, you can put extra information on the cover
% page as needed:
% \ifCLASSOPTIONpeerreview
% \begin{center} \bfseries EDICS Category: 3-BBND \end{center}
% \fi
%
% For peerreview papers, this IEEEtran command inserts a page break and
% creates the second title. It will be ignored for other modes.
\IEEEpeerreviewmaketitle


% Main content
\section{Εισαγωγή}
Η νοημοσύνη μπορεί να οριστεί ως η ικανότητα προσαρμογής της συμπεριφοράς ενός συστήματος σε ένα περιβάλλον που συνεχώς αλλάζει. Σύμφωνα με τον πατέρα της Τεχνητής Νοημοσύνης, Alan Turing, η μορφή και η εξωτερική εμφάνιση ενός συστήματος δεν έχει καμία σχέση με τη νοημοσύνη του.

Είναι γνωστό όμως ότι οι άνθρωποι παρουσιάζουν στοιχεία ευφυούς συμπεριφοράς. Οι άνθρωποι, είναι επίσης προϊόντα της φυσικής εξέλιξης και έτσι, με την μοντελοποίηση της εξελικτικής διαδικασίας, θα μπορούσε κανείς να φιλοδοξεί ότι θα δημιουργήσει μια ευφυή συμπεριφορά.

Οι εξελικτικοί αλγόριθμοι προσομοιώνουν την εξέλιξη στον υπολογιστή. Το αποτέλεσμα αυτής της προσομοίωσης είναι μια σειρά από αλγόριθμους βελτιστοποίησης, οι οποίοι βασίζονται συνήθως σε μια συλλογή από απλούς κανόνες. Η επαναληπτική βελτιστοποίηση βελτιώνει την ποιότητα των λύσεων μέχρι να βρεθεί η πιο βέλτιστη ή τουλάχιστον εφικτή λύση.

Ίσως να μην είναι ξεκάθαρη η νοημοσύνη της εξέλιξης, λόγο του ότι στη φύση χρειάζεται να περάσει αρκετός καιρός για να παρατηρηθούν οι σημαντικές αλλαγές σε κάποιο σύστημα. Ένας οργανισμός αναπτύσσει μια συμπεριφορά για κάποια στοιχεία του περιβάλλοντος που δεν έχει ξανασυναντήσει. Σε περίπτωση που επιζήσει μετά τις αλλαγές του περιβάλλοντος αυτή η συμπεριφορά κληρονομείται από τους απογόνους του και έτσι βγαίνει το συμπέρασμα ότι ένας οργανισμός έχει τη δυνατότητα της μάθησης καις της πρόβλεψης των αλλαγών στο περιβάλλον. Ολόκληρη η παραπάνω διαδικασία ίσως να φανεί πολύ αργή για τον άνθρωπο, αυτό όμως δεν ισχύει για τους υπολογιστές, καθώς η προσομοίωση της εξέλιξης δεν απαιτεί εκατομμύρια χρόνια.

Η εξελικτική προσέγγιση της μηχανικής μάθησης βασίζεται σε υπολογιστικά μοντέλα της φυσικής επιλογής και της γενετικής. Αυτή η προσέγγιση ονομάζεται εξελικτικοί αλγόριθμοι και συνδυάζει τεχνικές όπως οι γενετικοί αλγόριθμοι, οι εξελικτικές στρατηγικές και ο γενετικός προγραμματισμός. Όλες αυτές οι τεχνικές προσομοιώνουν την εξέλιξη με τη χρήση των διαδικασιών της επιλογής, της μετάλλαξης και του ανασυνδυασμού.

 
\section{Βιολογικό υπόβαθρο}

\subsection{Ο Κάρολος και η θεωρία εξέλιξης}
Όπως οι περισσότερες ανακαλύψεις τον ανθρώπων, έτσι και οι εξελικτικοί αλγόριθμοι είναι εμπνευσμένοι από την ίδια τη φύση. Η βάση για τη δημιουργία των εξελικτικών αλγορίθμων είναι η θεωρία της βιολογικής εξέλιξης η οποία προτάθηκε από τον Κάρολο Δαρβίνο στο πολύφημο έργο του "On the Origin of Species by Means of Natural Selection, or The Preservation of Favoured Races in the Struggle for Life".

Ο Δαρβίνος αναλύει την προσαρμοστικότητα των ζώων χρησιμοποιώντας την αρχή της φυσικής επιλογής. Δηλαδή στην ικανότητα των οργανισμών να προσαρμόζονται στο περιβάλλον και να αποκτούν απογόνους. Οι απόγονοι θα έχουν τα ίδια ή παρόμοια χαρακτηριστικά με τους προγόνους τους, έτσι ώστε να μπορέσουν να επιβιώσουν και να συνεχίσουν την αναπαραγωγή του είδους τους. Αυτός είναι ένας από τους κυριότερους μηχανισμούς της φυσικής εξέλιξης.

Η θεωρία του Δαρβίνου βασίζεται σε δεδομένα και στα συμπεράσματα που προκύπτουν από αυτά τα οποία συνόψισε ο βιολόγος Ernst Mayr ως εξής:
\begin{itemize}
  \item Κάθε είδος είναι γόνιμο σε τέτοιο βαθμό έτσι ώστε εάν επιβιώσουν όλοι οι απόγονοι τότε ο πληθυσμός τους θα αυξηθεί.
  \item Παρά τις περιοδικές διακυμάνσεις, οι πληθυσμοί εξακολουθούν να έχουν το ίδιο περίπου μέγεθος.
  \item Οι πόροι, όπως τα τρόφιμα και το νερό είναι περιορισμένοι και παραμένουν σταθεροί με την πάροδο του χρόνου. Έτσι προκύπτει ένας αγώνας για την επιβίωση.
  \item Τα άτομα σε έναν πληθυσμό ποικίλλουν σημαντικά το ένα από το άλλο.


  \item Τα άτομα που είναι λιγότερο κατάλληλα για το περιβάλλον έχουν τη μικρότερη πιθανότητα να επιζήσουν και να αναπαραχθούν, ενώ τα άτομα που έχουν όλα τα κατάλληλα χαρακτηριστικά για να επιβιώσουν στο περιβάλλον έχουν την μεγαλύτερη πιθανότητα να επιζήσουν και να αναπαραγάγουν απογόνους οι οποίοι κληρονομούν τα χαρακτηριστικά τους και έτσι δημιουργείται η διαδικασία της φυσικής επιλογής.

  \item Αυτή η σταδιακά αναπτυσσόμενη διαδικασία οδηγεί στην αλλαγή των πληθυσμών για την καλύτερη προσαρμοστικότητα και τελικά με τη συσσώρευση των διακυμάνσεων, καταλήγει στον σχηματισμό νέων ειδών.
\end{itemize}
Επιπρόσθετα, στο έργο του Δαρβίνου αναφέρονται οι διαφορές στα φυσικά και πνευματικά χαρακτηριστικά των οργανισμών \cite{Adamidis} όπως το το ύψος, το βάρος, η ευφυΐα, το χρώμα των ματιών και του τριχώματος. Δηλαδή αναλύονται οι αποκλίσεις ανάμεσα στους φαινότυπους των οργανισμών, οι οποίοι καθορίζουν τον τρόπο ανταπόκρισης και φυσικής ενσάρκωσης των γονέων και των παιδιών τους.

Όλες αυτές οι μεταλλάξεις είναι σημαντικές μόνο εάν δίνουν τη δυνατότητα στον οργανισμό να επιβιώσει σε συγκεκριμένες συνθήκες του περιβάλλοντος. Σε διαφορετική περίπτωση, οι μεταλλάξεις δεν έχουν καμία αξία αφού στο τέλος ο οργανισμός δεν έχει επιβιώνει και καταλήγει να εξαλείπτεται.

Στη περίπτωση όμως που οι συνθήκες του περιβάλλοντος είναι ευνοϊκές και οι οργανισμοί συνεχίζουν να αναπαράγονται, ο μόνος περιορισμός που υπάρχει είναι οι διαθεσιμότητα των πόρων. Για αυτό το λόγο, όταν οι πόροι είναι ανεπαρκής, επιβιώνουν μόνο οι οργανισμοί που τους εκμεταλλεύονται πιο αποδοτικά και αποτελεσματικά.

\subsection{Ανάλυση Βιολογικών Χαρακτηριστικών}
Από τη στιγμή που οι εξελικτικοί αλγόριθμοι βασίζονται σε βιολογικά χαρακτηριστικά της φύσης, είναι φανερό ότι για να τους κατανοήσει κανείς χρειάζεται να γνωρίζει κάποια βασικά βιολογικά στοιχεία.

Το σώμα όλων των οργανισμών αποτελείται από εκατομμύρια κύτταρα. Τα περισσότερα κύτταρα περιέχουν μια ολοκληρωμένη σειρά από γονίδια. Διαθέτουν λοιπόν χιλιάδες γονίδια. Τα γονίδια περιέχουν οδηγίες που ελέγχουν την ανάπτυξή και τον τρόπο που δουλεύει το σώμα. Είναι επίσης υπεύθυνα για πολλά φυσικά χαρακτηριστικά, όπως το χρώμα των ματιών, την ομάδα αίματος και το ύψος. Οι πιθανές τιμές κάθε γονιδίου ονομάζονται τιμές χαρακτηριστικών (alleles).

Στους βιολογικούς οργανισμούς \cite{Vlaxavas}, ένα χρωμόσωμα είναι ένα μεγάλο μόριο (ακολουθία) DNA και περιέχει έναν αριθμό γονιδίων. Κάθε γονίδιο έχει μια συγκεκριμένη θέση (locus) μέσα στο χρωμόσωμα. Στο πραγματικό DNA, το αλφάβητο έχει μήκος τέσσερα και αποτελείται από τα γράμματα A, G, T και C που αντιστοιχούν στα τέσσερα διαφορετικά νουκλεοτίδια (βάσεις) που το συνθέτουν (Adenine, Guanine, Thymine και Cytosine). Ένας οργανισμός μπορεί να έχει ένα ή περισσότερα χρωμοσώματα ενώ σε κάποιους οργανισμούς κάθε κύτταρο περιέχει δύο αντίγραφα για κάθε χρωμόσωμα. Για παράδειγμα, ο άνθρωπος έχει 23 ζεύγη χρωμοσωμάτων.

Στην κλασσική προσέγγιση των γενετικών αλγορίθμων, κάθε υποψήφια λύση αναπαριστάται με μία συμβολοσειρά (string) ενός πεπερασμένου αλφαβήτου. Συνήθως χρησιμοποιείται το δυαδικό αλφάβητο, οπότε οι συμβολοσειρές ονομάζονται και δυαδικές συμβολοσειρές (bit-strings). Ωστόσο, υπάρχουν περιπτώσεις γενετικών αλγορίθμων που χρησιμοποιούν πιο σύνθετες μορφές αναπαράστασης. Στα περισσότερα προβλήματα οι λύσεις περιγράφονται με μεταβλητές διαφόρων τύπων δεδομένων, επομένως η διαδικασία της κωδικοποίησης περιλαμβάνει τη μετατροπή των τιμών αυτών των μεταβλητών στις αντίστοιχες δυαδικές.

Κάθε γενετικός αλγόριθμος έχει ένα χρωμόσωμα, δηλαδή μια συμβολοσειρά με πεπερασμένο αριθμό χαρακτήρων. Τα επιμέρους τμήματα της συμβολοσειράς που κωδικοποιούν κάποιο χαρακτηριστικό, δηλαδή κάποια μεταβλητή, αντιπροσωπεύουν τα γονίδια.

Στη γενετική, το σύνολο των παραμέτρων που αναπαρίστανται από ένα συγκεκριμένο γονίδιο που μας ενδιαφέρει ή έναν αριθμό γονιδίων αναφέρεται σαν γονότυπος (genotype). Η συνολική εμφάνιση ενός οργανισμού ή εκδήλωση ενός χαρακτηριστικού ονομάζεται φαινότυπος και εξαρτάται άμεσα από το γονότυπο του.

Κάθε γονότυπος θα μπορούσε να αποτελεί μια πιθανή λύση στο πρόβλημα, η σημασία της οποίας καθορίζεται από το χρήστη. 
\section{Μοντέλα Εξελικτικών Αλγορίθμων}

Όπως αναφέρθηκε πιο πριν, οι εξελικτικοί αλγόριθμοι είναι εμπνευσμένοι από μηχανισμούς της βιολογικής εξέλιξης όπως είναι για παράδειγμα η αναπαραγωγή, η μετάλλαξη, ο ανασυνδυασμός και η φυσική επιλογή. \cite{_evolutionary_2014} Υπάρχουν διάφορα μοντέλα που μπορούν να χρησιμοποιηθούν για την επίλυση ενός προβλήματος με την χρήση εξελικτικών αλγορίθμων. Τα πιο διαδεδομένα από αυτά τα μοντέλα είναι οι Γενετικοί Αλγόριθμοι (Genetic Algorithms), ο Γενετικός Προγραμματισμός (Genetic Programming), οι Εξελικτικές Στρατηγικές (Evolution Strategies) και ο Εξελικτικός Προγραμματισμός (Evolutionary Programming). Οι τεχνικές αυτές δεν διαφέρουν πάρα πολύ μεταξύ τους, παρά μόνο στις λεπτομέρειες της υλοποίησής τους. \cite{Adamidis}

Η λειτουργία ενός εξελικτικού αλγορίθμου βασίζεται στην αρχικοποίηση του πληθυσμού και τη συνεχή εξέλιξη από γενιά σε γενιά με την λειτουργία της επιλογής, του ανασυνδυασμού και της μετάλλαξης. Αφού αρχικά γίνει η αρχικοποίηση του πληθυσμού, ακολουθεί η αξιολόγηση (evaluation) του με τη χρήση μιας συνάρτησης ποιότητας (fitness function) η οποία αποδίδει μια τιμή ποιότητας για κάθε άτομο του πληθυσμού στο συγκεκριμένο περιβάλλον.

Ειδικότερα, κατά την διάρκεια της επιλογής γίνεται η επιλογή γονέων και στη συνέχεια η επιβίωση. Στην επιλογή γονέων καθορίζεται ποια άτομα θα γίνουν γονείς και πόσους απογόνους θα αποκτήσουν. Οι απόγονοι δημιουργούνται με την ανταλλαγή πληροφοριών μεταξύ των γονέων και την περαιτέρω διαταραχή των απογόνων. Οι δύο παραπάνω διαδικασίες ονομάζονται ανασυνδυασμός και μετάλλαξη.

Στη συνέχεια αξιολογούνται οι απόγονοι με τη χρήση της συνάρτησης ποιότητας και στο τέλος επιλέγονται τα άτομα που θα επιβιώσουν και θα περάσουν στην επόμενη γενιά. Αυτή η διαδικασία ονομάζεται εξελικτικός κύκλος (evolutionary cycle). Στον αλγόριθμο \ref{algo_evAlg} παρουσιάζεται αυτή η διαδικασία σε μορφή ψευδοκώδικα.

%\begin{figure}
%    \centering
%    \def\svgwidth{2.1in}
%    \input{./figures/EvolutionaryCycle.pdf_tex}
%    \caption{Εξελικτικός κύκλος}
%    \label{fig_evCycl}
%\end{figure}

\begin{algorithm}[!t]
    \caption{Γενική μορφή ενός εξελικτικού αλγόριθμου}
    \label{algo_evAlg}
    \KwIn{populationSize, maxEvolutions}
    \KwOut{The chromosome with the best fitness value}
    $population \gets randomPopulation(populationSize)$\;
    $evolutions \gets 0$\;
    \While{$evolutions < maxEvolutions$}{
        $parents \gets selectParents(population)$\;
        $population \gets recombine(parents)$\;
        $population \gets mutate(population)$\;
        $evolutions \gets evolutions + 1$\;
    }
    \Return{$getBestChromosome(population)$}\;
\end{algorithm}

Το μοντέλο που χρησιμοποιείται πιο συχνά για την επίλυση προβλημάτων με την χρήση εξελικτικών αλγόριθμων είναι οι γενετικοί αλγόριθμοι, οι οποίοι αναλύονται παρακάτω.

\subsection{Γενετικοί Αλγόριθμοι}

Οι γενετικοί αλγόριθμοι εφευρέθηκαν την δεκαετία του 1960 από τον John Holland, και αναπτύχθηκαν περισσότερο στο Πανεπιστήμιο του Μίσιγκαν από τον ίδιο, τους φοιτητές του και τους συναδέλφους του. Ο στόχος του Holland ήταν να μελετήσει το φαινόμενο της προσαρμογής όπως αυτό παρατηρείται στη φύση, και όχι να επιλύσει συγκεκριμένα προβλήματα. Ήθελε με αυτόν τον τρόπο να αναπτύξει μεθόδους για την εισαγωγή των μηχανισμών της φυσικής επιλογής σε υπολογιστικά συστήματα. \cite{Melanie1999}

O Holland ασχολήθηκε με αλγορίθμους που χειρίζονται συμβολοσειρές από δυαδικά ψηφία. Θεωρούσε αυτούς τους αλγόριθμους ως μια αφηρημένη μορφή της φυσικής εξέλιξης. Οι γενετικοί αλγόριθμοι του μπορούσαν να αναπαρασταθούν με μια σειρά από διαδικαστικά βήματα για τη μετάβαση από τον ένα πληθυσμό τεχνητών χρωμοσωμάτων σε έναν καινούργιο.

Στην ουσία, οι γενετικοί αλγόριθμοι είναι μια κλάση από στοχαστικούς αλγόριθμους αναζήτησης οι οποίοι είναι βασισμένοι στη βιολογική εξέλιξη. Ακολουθούν μια επαναληπτική διαδικασία και κάθε επανάληψη ονομάζετα γενιά.

Στη φύση υπάρχει η δυνατότητα της προσαρμογής και μάθησης χωρίς την καθοδήγηση από κάποιον τρίτο. Το ίδιο ισχύει και για τους γενετικούς αλγορίθμους. Οι μηχανισμοί που συνδέουν τους γενετικούς αλγορίθμους με το πρόβλημα είναι η αναπαράσταση και η αξιολόγηση.

\subsection{Τρόποι Αναπαράστασης}

Ο πιο συνηθισμένος τρόπος αναπαράστασης των χρωμοσωμάτων στους γενετικούς αλγόριθμους είναι η σταθερού-μήκους συμβολοσειρά δυαδικών ψηφίων (bit string). Η χρήση των bit string διευκολύνει την διαδικασία εύρεσης απογόνων.

\begin{figure}[!t]
    \renewcommand{\arraystretch}{1.3}
    \label{fig_bit_string}
    \centering
    \begin{tabular}{c|c|c|c|c|c|c|c}
        \hline
        \ldots & 0 & 0 & 1 & 0 & 1 & 1 & \ldots\\
        \hline
    \end{tabular}
    \caption{Παράδειγμα bit string}
\end{figure}

Ένα πρόβλημα που παρουσιάζουν τα bit string είναι ο λεγόμενος «γκρεμός Hamming». Για παράδειγμα, ο αριθμός $255$ στο δυαδικό σύστημα γράφεται ως $011111111$ και ο αριθμός $256$ ως $100000000$. Για την αλλαγή από τον έναν αριθμό στον άλλον πρέπει να αλλάξουν $9$ bits, που σημαίνει ότι είναι πολύ πιθανό να πρέπει να βρεθούν πολλοί απόγονοι μέχρι το χρωμόσωμα να φτάσει σε μια πιο βέλτιστη λύση. Αυτό το πρόβλημα μπορεί να αντιμετωπιστεί κωδικοποιώντας τους αριθμούς με τον κώδικα Gray. Ό κώδικας Gray είναι ένα δυαδικό σύστημα όπου δύο διαδοχικοί αριθμοί διαφέρουν μόνο κατά ένα bit. \cite{Lehre} Στον πίνακα \ref{table_gray_code} φαίνεται η κωδικοποίηση Gray για ένα bit string μήκους 3 bit.

\begin{table}[!t]
    \renewcommand{\arraystretch}{1.3}
    \caption{Κωδικοποίηση Gray 3 bit}
    \label{table_gray_code}
    \centering
    \begin{tabular}{c|c|c}
        \hline
        \bfseries Δεκαδικό & \bfseries Δυαδικό & \bfseries Κώδικας Gray\\
        \hline\hline
        0 & 000 & 000\\
        1 & 001 & 001\\
        2 & 010 & 011\\
        3 & 011 & 010\\
        4 & 100 & 110\\
        5 & 101 & 111\\
        6 & 110 & 101\\
        7 & 111 & 100\\
        \hline
    \end{tabular}
\end{table}

Άλλοι τρόποι αναπαράστασης που μπορούν να χρησιμοποιηθούν ανάλογα με το πρόβλημα είναι οι μεταθέσεις και η δεντρική δομή. Οι μεταθέσεις χρησιμοποιούνται συνήθως σε προβλήματα ταξινόμησης, όπως για παράδειγμα το γνωστό πρόβλημα του περιπλανώμενου πωλητή (το κάθε χρωμόσωμα δίνει την σειρά που ο πωλητής θα επισκεφθεί τις πόλεις). \cite{Obitko}

\subsection{Δημιουργία Πληθυσμού}

Στη φάση της δημιουργίας του αρχικού πληθυσμού, συνήθως δημιουργούνται Ν τυχαία σταθερού-μήκους bit strings, όπου Ν είναι το μέγεθος του πληθυσμού που έχει οριστεί. Υπάρχουν έρευνες που αποδεικνύουν ότι όσο μεγαλύτερο το μέγεθος του πληθυσμού, τόσο μεγαλύτερη η ακρίβεια του γενετικού αλγόριθμου, αλλά και τόσες περισσότερες γενιές χρειάζονται για να αυξηθεί η σύγκλιση. \cite{Gotshall2008}

\subsection{Αξιολόγηση Πληθυσμού}

%Αφού δημιουργηθεί ο αρχικός πληθυσμός, θα πρέπει να επιλεχθούν τα χρωμοσώματα που θα γίνουν γονείς και θα παράγουν τα χρωμοσώματα-παιδιά που θα χρησιμοποιηθούν για τον επόμενο κύκλο του γενετικού αλγόριθμου. Είναι επιθυμητό τα χρωμοσώματα που δίνουν καλύτερη λύση στο πρόβλημα να έχουν περισσότερες πιθανότητες να επιλεχθούν.

Αφού δημιουργηθεί ο αρχικός πληθυσμός, θα πρέπει να αξιολογηθεί το πόσο καλή είναι η λύση που δίνει το κάθε χρωμόσωμα. Για τον έλεγχο της ποιότητας ενός χρωμοσώματος, χρησιμοποιείται η λεγόμενη συνάρτηση καταλληλότητας (fitness function) ή αλλιώς συνάρτηση αξιολόγησης.

Η συνάρτηση καταλληλότητας δέχεται σαν είσοδο ένα χρωμόσωμα, και επιστρέφει έναν αριθμό ο οποίος υποδηλώνει το πόσο καλό είναι το χρωμόσωμα. Εάν επιλύεται πρόβλημα μεγιστοποίησης, τότε όσο πιο μεγάλη τιμή επιστρέφει η συνάρτηση καταλληλότητας για το χρωμόσωμα, τόσο πιο καλή λύση αναπαριστά το χρωμόσωμα. Αντιθέτως, εάν επιλύεται πρόβλημα ελαχιστοποίησης, τότε όσο πιο χαμηλή τιμή καταλληλότητας έχει ένα χρωμόσωμα, τόσο καλύτερη λύση αναπαριστά.

\subsection{Επιλογή Γονέων}

Μετά την αξιολόγηση του πληθυσμού, πρέπει να επιλεχθούν τα χρωμοσώματα που θα γίνουν γονείς και θα παράγουν τα χρωμοσώματα-παιδιά που θα χρησιμοποιηθούν για τον επόμενο κύκλο του γενετικού αλγόριθμου. Είναι επιθυμητό τα χρωμοσώματα που δίνουν καλύτερη λύση στο πρόβλημα να έχουν περισσότερες πιθανότητες να επιλεχθούν. Υπάρχουν διάφορες τεχνικές για την επιλογή των γονέων. Οι πιο διαδεδομένες από αυτές αναλύονται παρακάτω.

\subsubsection{Επιλογή Ρουλέτας}

Μία από αυτές τις τεχνικές είναι η επιλογή ρουλέτας (roulette wheel selection) που αναφέρθηκε παραπάνω, όπου τα χρωμοσώματα καταλαμβάνουν χώρο στη «ρουλέτα» ανάλογα με την ποιότητα τους (οπότε υπάρχει μεγαλύτερη πιθανότητα να «πέσει η μπίλια» στα καλύτερα χρωμοσώματα).

\begin{figure*}[!t]
    \centering
    \begin{subtable}[t!]{\linewidth}
        \centering
        \begin{tabular}{@{}crrrr@{}}
            \toprule
            Α/Α Χρωμοσώματος & Καταλληλότητα & Συνολικό ποσοστό & Κατάταξη & Ποσοστό κατάταξης \\ \midrule
            1                & 40            & 14.8\%           & 4        & 19\%              \\
            2                & 110           & 40.7\%           & 6        & 28.5\%            \\
            3                & 30            & 11.1\%           & 3        & 14.3\%            \\
            4                & 25            & 9.3\%            & 2        & 9.6\%             \\
            5                & 45            & 16.7\%           & 5        & 23.8\%            \\
            6                & 20            & 7.4\%            & 1        & 4.8\%             \\ \midrule
            Σύνολο           & 270           & 100\%            &          & 100\%             \\ \bottomrule
        \end{tabular}
        \caption{Πληθυσμός}
    \end{subtable}
    \\
    \begin{subfigure}[t!]{.4\linewidth}
        \centering
        \begin{tikzpicture}
            \pie[color={black!10, black!35, black!30, black!25, black!20, black!15}, rotate=45, radius=2]{14.8/1, 7.4/6, 16.7/5, 9.3/4, 11.1/3, 40.7/2}
        \end{tikzpicture}
        \caption{Ποσοστά καταλληλότητας}
    \end{subfigure}
    ~
    \begin{subfigure}[t!]{.4\linewidth}
        \centering
        \begin{tikzpicture}
            \pie[color={black!10, black!35, black!30, black!25, black!20, black!15}, rotate=45, radius=2]{19/1, 4.8/6, 23.8/5, 9.6/4, 14.3/3, 28.5/2}
        \end{tikzpicture}
        \caption{Ποσοστά κατάταξης}
    \end{subfigure}

    \caption{Παράδειγμα επιλογής ρουλέτας}
    \label{fig_rouletteWheel}
\end{figure*}

\begin{algorithm}[!t]
    \caption{Ψευδοκώδικας επιλογής ρουλέτας}
    \label{algo_rouletteWheel}
    \KwIn{population}
    \KwOut{The selected chromosomes}
    $fitnessSum \gets 0$\;
    \ForEach{$x$ in $population$}{
        $fitnessSum \gets fitnessSum + fitness(x)$\;
    }
    \While{$newPopulation$ is not full}{
        $randNum \gets randBetween(0, fitnessSum)$\;
        $relFitness \gets 0$\;
        $i \gets 0$\;
        \While{$randNum > relFitness$}{
            $relFit \gets relFit + fitness(population_i)$\;
            $i \gets i + 1$\;
        }
        $newPopulation.add(population_{i-1})$\;
    }
    \Return{$newPopulation$}\;
\end{algorithm}

\subsubsection{Επιλογή Κατάταξης}

Η επιλογή κατάταξης (rank selec\-tion) μοιάζει πολύ με την επιλογή ρουλέτας, μόνο που η πιθανότητα επιλογής ενός χρωμοσώματος είναι ανάλογη της σχετικής καταλληλότητας (κατάταξη) αντί τις απόλυτης καταλληλότητας. Για έναν πληθυσμό με $N$ άτομα, το καλύτερο άτομο έχει κατάταξη $N$, το επόμενο καλύτερο έχει κατάταξη $N-1$, κ.ο.κ. \cite{manlio2014}

Αυτό σημαίνει ότι δεν έχει σημασία το πόσο πολύ διαφέρει η καταλληλότητα ανάμεσα σε δύο άτομα, αλλά το πόσο διαφέρει η κατάταξή τους. Για παράδειγμα, στο σχήμα \ref{fig_rouletteWheel} φαίνεται ότι ενώ το χρωμόσωμα 2 έχει ποσοστό $40.7\%$ να επιλεγεί με την μέθοδο επιλογής ρουλέτας, στην μέθοδο επιλογής κατάταξης έχει μόνο $28.5\%$ πιθανότητα να επιλεχτεί.

\subsubsection{Επιλογή Τουρνουά}

Μια άλλη τεχνική ονομάζεται επιλογή τουρνουά (tournament selection), και λειτουργεί επιλέγοντας ένα τυχαίο υποσύνολο k χρωμοσωμάτων από τον αρχικό πληθυσμό, και επιλέγοντας το καλύτερο χρωμόσωμα από αυτό το υποσύνολο.

\subsection{Αναπαραγωγή των Γονέων}

Μετά την χρήση μιας από τις παραπάνω μεθόδους για την επιλογή των χρωμοσωμάτων που θα χρησιμοποιηθούν ως γονείς στην αναπαραγωγή, πρέπει να εφαρμοστεί ένας ή περισσότεροι τελεστές για την δημιουργία των χρωμοσωμάτων-παιδιών. Μερικοί από τους τελεστές που χρησιμοποιούνται συχνότερα είναι η διασταύρωση (crossover) και η μετάλλαξη.

\subsubsection{Διασταύρωση ενός σημείου}

Στη διασταύρωση ενός σημείου (one-point crossover) επιλέγεται τυχαία ένα σημείο στο χρωμόσωμα των γονέων, και τα δεδομένα μετά από αυτό το σημείο ανταλλάσσονται μεταξύ των δύο γονέων (σχήμα \ref{fig_opc}). Για παράδειγμα, έστω ότι τα χρωμοσώματα $0110100$ και $1100110$ είναι οι δύο γονείς, και το σημείο διασταύρωσης είναι το $4$. Τότε τα καινούργια χρωμοσώματα θα είναι τα $0110110$ και $1100100$.

\begin{figure}[!t]
    \centering
    \def\svgwidth{2.5in}
    \input{./figures/OnePointCrossover.pdf_tex}
    \caption{Διασταύρωση ενός σημείου \cite{opc_fig}}
    \label{fig_opc}
\end{figure}

\subsubsection{Διασταύρωση δύο σημείων}

Στη διασταύρωση δύο σημείων (two-point crossover), επιλέγονται τυχαία δύο σημεία στο χρωμόσωμα των γονέων, και τα δεδομένα που βρίσκονται ανάμεσα στα δύο σημεία ανταλλάσσονται μεταξύ των δύο γονέων (σχήμα \ref{fig_tpc}). Για παράδειγμα, έστω πάλι τα χρωμοσώματα $0110100$ και $1100110$, και τα δύο τυχαία σημεία είναι το 2 και το 5. Τότε, τα παιδιά που θα παραχθούν θα είναι τα $0100100$ και $1110110$.

\begin{figure}[!t]
    \centering
    \def\svgwidth{2.5in}
    \input{./figures/TwoPointCrossover.pdf_tex}
    \caption{Διασταύρωση δύο σημείων \cite{tpc_fig}}
    \label{fig_tpc}
\end{figure}

\subsubsection{Μετάλλαξη}

Στη μετάλλαξη (mutation) επιλέγεται τυχαία ένα bit από το χρωμόσωμα, και αυτό αντιστρέφεται (δηλαδή από $0$ γίνεται $1$, και από $1$ γίνεται $0$, σχήμα \ref{fig_mutation}). Αυτός ο τελεστής πρέπει να έχει μικρή πιθανότητα να εφαρμοστεί (κοντά στο $1\%$), αλλιώς υπάρχει κίνδυνος η αναζήτηση να μετατραπεί σε τυχαία αναζήτηση. \cite{zotero-PNQJ3TKI}

\begin{figure}[!t]
    \centering
    \begin{subtable}[b]{.4\linewidth}
        \centering
        \begin{tabular}{|c|c|c|c|}
            \hline
            1 & 0 & \cellcolor{gray!25}0 & 0\\
            \hline
        \end{tabular}
        \caption{Πριν}
        \label{fig_beforeMutation}
    \end{subtable}
    ~
    \begin{subtable}[b]{.3\linewidth}
        \centering
        \begin{tabular}{|c|c|c|c|}
            \hline
            1 & 0 & \cellcolor{gray!25}1 & 0\\
            \hline
        \end{tabular}
        \caption{Μετά}
        \label{fig_afterMutation}
    \end{subtable}
    \caption{Μετάλλαξη ενός χρωμοσώματος}
    \label{fig_mutation}
\end{figure}

\subsection{Εξελικτικές Στρατηγικές}
Μια άλλη προσέγγιση για τη προσομοίωση της φυσικής εξέλιξης προτάθηκε στη Γερμανία στις αρχές τις δεκαετίας του '60. Σε αντίθεση με τους γενετικούς αλγόριθμους, αυτή η μέθοδος η οποία ονομάστηκε εξελικτικές στρατηγικές, σχεδιάστηκε για την επίλυση τεχνικών προβλημάτων βελτιστοποίησης.

Το 1963 δύο φοιτητές του Τεχνικού Πανεπιστημίου του Βερολίνου, ο Ingo Rechenberg και ο Hans-Paul Schwefel, αναζητούσαν τα καταλληλότερα σχήματα σωμάτων μέσα από τα οποία θα υπάρχει ροή νερού ή αέρα. Για την μελέτη τους χρησιμοποίησαν μια αεροδυναμική σήραγγα. Επειδή η διαδικασία της εξαγωγής των πειραμάτων ήταν επίπονη, αποφάσισαν να εφαρμόσουν τυχαίες αλλαγές στις παραμέτρους που καθόριζαν το σχήμα του σώματος, ακολουθώντας το παράδειγμα της φυσικής μετάλλαξης. Ως αποτέλεσμα, γεννήθηκαν οι εξελικτικές στρατηγικές.

Οι εξελικτικές στρατηγικές αναπτύχθηκαν ως μια εναλλακτική της διαισθητικής ικανότητας των μηχανικών. Μέχρι πρόσφατα, η χρήση τους αφορούσε τα τεχνικά προβλήματα βελτιστοποίησης που δεν διέθεταν κάποια αναλυτική αντικειμενική συνάρτηση και ούτε κάποια συμβατική μέθοδος βελτιστοποίησης, με αποτέλεσμα οι μηχανικοί να αναγκάζονται να ακολουθήσουν τη διαίσθησή τους.

Σε αντίθεση με τους γενετικούς αλγορίθμους, οι εξελικτικές στρατηγικές χρησιμοποιούν μόνο τελεστές μετάλλαξης και δεν υπάρχει η ανάγκη για την κωδικοποίηση του προβλήματος.

Στην απλή της μορφή, η οποία ονομάζεται (1+1)-εξελικτική στρατηγική, κάθε γονέας δημιουργεί έναν απόγονο για κάθε γενιά με την εφαρμογή της κανονικά κατανεμημένης μετάλλαξης.
\subsection{Γενετικός Προγραμματισμός}

Ένα από τα κυριότερα προβλήματα στην επιστήμη των υπολογιστών είναι η δημιουργία υπολογιστών οι οποίοι θα είναι σε θέση να λύσουν προβλήματα για τα οποία όμως δεν έχουν προγραμματισθεί ρητά. Ο γενετικός προγραμματισμός προσφέρει μια λύση σε αυτό το πρόβλημα μέσα από την εξέλιξη των υπολογιστικών προγραμμάτων χρησιμοποιώντας μεθόδους της φυσικής εξέλιξης.

Στη πραγματικότητα, ο γενετικός προγραμματισμός είναι μια επέκταση του συμβατικού γενετικού αλγορίθμου, όμως αυτή τη φορά ο στόχος δεν είναι μόνο η εξέλιξη μια δυαδικής συμβολοσειράς κάποιου προβλήματος, αλλά η εξέλιξη του ίδιου του αλγορίθμου που λύνει το πρόβλημα.

Ο γενετικός προγραμματισμός είναι μια πρόσφατη ανάπτυξη στο τομέα των εξελικτικών αλγορίθμων. Ενισχύθηκε σημαντικά στη δεκαετία του '90 από τον John Koza.
Σύμφωνα με αυτόν, ο γενετικός προγραμματισμός αναζητά σε ένα χώρο από πιθανά προγράμματα εκείνο το πρόγραμμα που αρμόζει περισσότερο για την λύση ενός συγκεκριμένου προβλήματος.

Κάθε πρόγραμμα είναι μια ακολουθία από λειτουργίες εφαρμοσμένες σε παραμέτρους. Διαφορετικές γλώσσες προγραμματισμού μπορεί να περιέχουν διαφορετικούς τύπους, τελεστές και συντακτικούς περιορισμούς. Εφόσον ο γενετικός προγραμματισμός παραποιεί τα προγράμματα με την εφαρμογή γενετικών τελεστών, η γλώσσα προγραμματισμού που χρησιμοποιείται θα πρέπει να χειρίζεται το πρόγραμμα σαν δεδομένα και όλα τα δεδομένα που παράγονται θα πρέπει να έχουν την δυνατότητα να εκτελεστούν. Για αυτούς τους λόγους επιλέχθηκε η LISP ως κύρια γλώσσα προγραμματισμού για τους γενετικούς αλγορίθμους.

Πριν την εφαρμογή του γενετικού προγραμματισμού, θα πρέπει να πραγματοποιηθούν τα εξής προκαταρκτικά βήματα:

\begin{enumerate}
  \item Καθορισμός του αριθμού των τερματικών
  \item Επιλογή του συνόλου των παραγόντων συναρτήσεων
  \item Καθορισμός της συνάρτησης καταλληλότητας
  \item Καθορισμός των παραμέτρων για τον έλεγχο της εκτέλεσης
  \item Επιλογής της μεθόδου εξαγωγής αποτελέσματος για την εκτέλεση
\end{enumerate}

Μόλις ολοκληρωθούν τα 5 παραπάνω βήματα, μπορεί να αρχίσει η εκτέλεση, η οποία ξεκινά με τη δημιουργία τυχαίου αρχικού πληθυσμού από προγράμματα. Κάθε πρόγραμμα αποτελείται από μεθόδους και τερματικά.

Στον αρχικό πληθυσμό, όλα τα προγράμματα συνήθως έχουν χαμηλή ποιότητα, αλλά κάποια μεμονωμένα είναι πιο ποιοτικά από άλλα. Ακριβώς όπως ένα ποιοτικότερο χρωμόσωμα έχει μεγαλύτερη πιθανότητα να επιλεγεί για να αναπαράγει απογόνους, έτσι και ένα ποιοτικότερο πρόγραμμα, με βάση τη συνάρτηση καταλληλότητας, έχει μεγαλύτερη πιθανότητα να επιβιώσει αντιγράφοντας τον εαυτό του στην επόμενη γενιά.

Στον γενετικό προγραμματισμό, ο τελεστής της διασταύρωσης εφαρμόζεται σε δύο πρόγραμμα τα οποία επιλέγονται με βάση την καταλληλότητα τους. Τα προγράμματα μπορούν να έχουν διαφορετικό μέγεθος και σχήμα. Τα δύο προγράμματα που παράγονται ως απόγονοι, συνθέτονται με τον ανασυνδυασμό τυχαίων σημείων των προγόνων τους. Ο τελεστής παράγει έγκυρα προγράμματα ως απογόνους ανεξαρτήτως την επιλογή των σημείων διασταύρωσης.

Ο τελεστής μετάλλαξης μπορεί τυχαία να αλλάξει οποιαδήποτε μέθοδο ή οποιοδήποτε τερματικό σε ένα πρόγραμμα. Μια μέθοδος όμως μπορεί να αντικατασταθεί μόνο από μία άλλη μέθοδο και ένα τερματικό από ένα άλλο τερματικό.

\section{Επίλυση}

\subsection{Ανάλυση Βιβλιοθήκης JGAP}

Μια βιβλιοθήκη που επιτρέπει την ανάπτυξη εξελικτικών αλγορίθμων είναι η JGAP. Η βιβλιοθήκη αυτή έχει αναπτυχθεί στη γλώσσα Java και υποστηρίζει μεγάλη προσαρμοστικότητα χωρίς όμως αυτό να σημαίνει ότι είναι δύσκολη στη χρήση. Στην πραγματικότητα συμβαίνει το ακριβώς αντίθετο.

Παρέχει όλους τους βασικούς μηχανισμούς οι οποίοι μπορούν να χρησιμοποιηθούν για την εφαρμογή των εξελικτικών αρχών στις λύσεις των προβλημάτων. \cite{Meffert}

\begin{figure}[!t]
    \centering
    \begin{tikzpicture}
        \begin{axis}[
            xlabel={Πιθανότητα ανασυνδυασμού},
            ylabel={Καταλληλότητα},
            ymin = 2300,
            legend cell align = left,
            legend pos = south east,
            font = \footnotesize]

            \addplot +[mark=none] table [x=a, y=b, col sep=comma] {./figures/crossoverRateFitness.csv};
            \addlegendentry{πληθ. = 50, γεν. = 5}
            \addplot +[mark=none] table [x=a, y=c, col sep=comma] {./figures/crossoverRateFitness.csv};
            \addlegendentry{πληθ. = 50, γεν. = 10}
            \addplot +[mark=none] table [x=a, y=d, col sep=comma] {./figures/crossoverRateFitness.csv};
            \addlegendentry{πληθ. = 20, γεν. = 5}
            \addplot +[mark=none] table [x=a, y=e, col sep=comma] {./figures/crossoverRateFitness.csv};
            \addlegendentry{πληθ. = 20, γεν. = 10}
        \end{axis}
    \end{tikzpicture}
    \caption{Μέσος όρος καλύτερου fitness value καθώς μεγαλώνει το ποσοστό ανασυνδυασμού}
    \label{fig_crossover}
\end{figure}

\begin{figure}[!t]
    \centering
    \begin{tikzpicture}
        \begin{axis}[
            xlabel={Αριθμός γενιάς},
            ylabel={Καταλληλότητα},
            %ymin = 2300,
            legend cell align = left,
            legend pos = south east,
            font = \footnotesize]

            \addplot +[mark=none] table [x=gen, y=pop20, col sep=comma] {./figures/averageFitness.csv};
            \addlegendentry{πληθ. = 20}
            \addplot +[mark=none] table [x=gen, y=pop50, col sep=comma] {./figures/averageFitness.csv};
            \addlegendentry{πληθ. = 50}
            \addplot +[mark=none] table [x=gen, y=pop100, col sep=comma] {./figures/averageFitness.csv};
            \addlegendentry{πληθ. = 100}
            \addplot +[mark=none] table [x=gen, y=pop1000, col sep=comma] {./figures/averageFitness.csv};
            \addlegendentry{πληθ. = 1000}
        \end{axis}
    \end{tikzpicture}
    \caption{Μέσος όρος καλύτερου fitness value ανά γενιά}
    \label{fig_avgFitness}
\end{figure}

\begin{figure}[!t]
    \centering
    \begin{tikzpicture}
    	\begin{axis}[
            xlabel={x},
            ylabel={y},
            width = 0.8\columnwidth,
            %height = 0.8\columnwidth,
            colorbar,
            legend cell align = left,
            legend pos = north west,
            %colormap = {whiteblack}{gray(0cm) = (1); gray(1cm) = (0)},
            colorbar style = {title=Αριθμός Σφηκών, /tikz/.cd},
            font = \footnotesize]

            \addplot +[scatter, only marks, point meta=explicit] table [meta=wasps, col sep=comma] {./figures/waspNests.csv};
            \addlegendentry{Σφηκοφωλιά}
        \end{axis}
    \end{tikzpicture}
    \caption{Χάρτης σφηκοφωλιών}
    \label{fig_waspNestsMap}
\end{figure} 

\section{Conclusion}
The συμπέρασμα goes here.

% conference papers do not normally have an appendix

% use section* for acknowledgement
\section*{Acknowledgment}

The authors would like to thank...

% references section

\begingroup
\raggedright
\printbibliography
\endgroup
% that's all folks
\end{document}


