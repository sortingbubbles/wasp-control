\section{Επίλυση}

\subsection{Ανάλυση Βιβλιοθήκης JGAP}

Η υλοποίηση των εξελικτικών αλγορίθμων απαιτεί πολύ χρόνο και κόπο. Για αυτό το λόγο έχουν αναπτυχθεί διάφορες βιβλιοθήκες οι οποίες παρέχουν όλους τους βασικούς  μηχανισμούς για την υλοποίηση ενός εξελικτικού αλγορίθμου.

Μια τέτοια βιβλιοθήκη που επιτρέπει την ανάπτυξη εξελικτικών αλγορίθμων είναι η JGAP. Η βιβλιοθήκη αυτή έχει αναπτυχθεί στη γλώσσα Java και υποστηρίζει μεγάλη προσαρμοστικότητα χωρίς όμως αυτό να σημαίνει ότι είναι δύσκολη στη χρήση. Στην πραγματικότητα συμβαίνει το ακριβώς αντίθετο. 

Παρέχει όλες τις απαραίτητες κλάσεις και μεθόδους οι οποίες μπορούν να χρησιμοποιηθούν για την εφαρμογή των εξελικτικών αρχών στις λύσεις των προβλημάτων. \cite{Meffert}

Για να αναπτυχθεί ένας γενετικός αλγόριθμος με τη χρήση της βιβλιοθήκης JGAP, θα πρέπει να πραγματοποιηθούν τα εξής βήματα: \cite{zotero-R7UV9N25}

\begin{enumerate}
  \item Σχεδιασμός χρωμοσώματος
  \item Υλοποίηση συνάρτησης καταλληλότητας
  \item Ορισμός των παραμέτρων 
  \item Δημιουργία ενός πληθυσμού από υποψήφιες λύσεις
  \item Εξέλιξη του πληθυσμού
\end{enumerate}

\subsubsection{Σχεδιασμός Χρωμοσώματος}

Όπως αναφέρθηκε, το χρωμόσωμα αναπαριστά μια υποψήφια λύση ενός προβλήματος και αποτελείται από πολλαπλά γονίδια. Τα γονίδια στην βιβλιοθήκη JGAP αναπαριστούν διακριτά χαρακτηριστικά της λύσης. Για αυτό το λόγο το πρώτο βήμα αφορά τον ορισμό του χρωμοσώματος. Δηλαδή στην ουσία γίνεται ο ορισμός των γονιδίων (genes).

Αυτό γίνεται με την υλοποίηση της διεπαφής Gene η οποία αντιπροσωπεύει ένα γονίδιο χρωμοσώματος. Τέτοιες κλάσεις είναι η IntegerGene, DoubleGene και άλλες. 

Για την δημιουργία του χρωμοσώματος, τα γονίδια τοποθετούνται σε ένα πίνακα και στη συνέχεια δημιουργείται ένα αντικείμενο τύπου Chromosome το οποίο δέχεται σαν όρισμα το πίνακα με τα γονίδια.

\subsubsection{Υλοποίηση Συνάρτησης Καταλληλότητας}

Για την αξιολόγηση των χρωμοσωμάτων είναι αναγκαία η χρήση μιας συνάρτησης καταλληλότητας. Στη βιβλιοθήκη JGAP, αυτό επιτυχγάνεται με την δημιουργίας μιας κλάσης που επεκτείνει την ήδη υπάρχουσα αφηρημένη κλάση FitnessFunction.

Σε αυτήν θα πρέπει να υλοποιηθεί η μέθοδος evaluate, η οποία δέχεται σαν όρισμα ένα χρωμόσωμα και αναλόγως επιστρέφει την τιμή της καταλληλότητας του.






\begin{figure}[!t]
    \centering
    \begin{tikzpicture}
        \begin{axis}[
            xlabel={Πιθανότητα ανασυνδυασμού},
            ylabel={Καταλληλότητα},
            ymin = 2300,
            legend cell align = left,
            legend pos = south east,
            font = \footnotesize]

            \addplot +[mark=none] table [x=a, y=b, col sep=comma] {./figures/crossoverRateFitness.csv};
            \addlegendentry{πληθ. = 50, γεν. = 5}
            \addplot +[mark=none] table [x=a, y=c, col sep=comma] {./figures/crossoverRateFitness.csv};
            \addlegendentry{πληθ. = 50, γεν. = 10}
            \addplot +[mark=none] table [x=a, y=d, col sep=comma] {./figures/crossoverRateFitness.csv};
            \addlegendentry{πληθ. = 20, γεν. = 5}
            \addplot +[mark=none] table [x=a, y=e, col sep=comma] {./figures/crossoverRateFitness.csv};
            \addlegendentry{πληθ. = 20, γεν. = 10}
        \end{axis}
    \end{tikzpicture}
    \caption{Μέσος όρος καλύτερου fitness value καθώς μεγαλώνει το ποσοστό ανασυνδυασμού}
    \label{fig_crossover}
\end{figure}

\begin{figure}[!t]
    \centering
    \begin{tikzpicture}
        \begin{axis}[
            xlabel={Αριθμός γενιάς},
            ylabel={Καταλληλότητα},
            %ymin = 2300,
            legend cell align = left,
            legend pos = south east,
            font = \footnotesize]

            \addplot +[mark=none] table [x=gen, y=pop20, col sep=comma] {./figures/averageFitness.csv};
            \addlegendentry{πληθ. = 20}
            \addplot +[mark=none] table [x=gen, y=pop50, col sep=comma] {./figures/averageFitness.csv};
            \addlegendentry{πληθ. = 50}
            \addplot +[mark=none] table [x=gen, y=pop100, col sep=comma] {./figures/averageFitness.csv};
            \addlegendentry{πληθ. = 100}
            \addplot +[mark=none] table [x=gen, y=pop1000, col sep=comma] {./figures/averageFitness.csv};
            \addlegendentry{πληθ. = 1000}
        \end{axis}
    \end{tikzpicture}
    \caption{Μέσος όρος καλύτερου fitness value ανά γενιά}
    \label{fig_avgFitness}
\end{figure}

\begin{figure}[!t]
    \centering
    \begin{tikzpicture}
        \begin{axis}[
            xlabel={Αριθμός γενιάς},
            ylabel={Καταλληλότητα},
            %ymin = 2300,
            legend cell align = left,
            legend pos = south east,
            font = \footnotesize]

            \addplot +[mark=none] table [x=gen, y=bestchrom, col sep=comma] {./figures/selectors.csv};
            \addlegendentry{BestChromosomesSelector}
            \addplot +[mark=none] table [x=gen, y=roulette, col sep=comma] {./figures/selectors.csv};
            \addlegendentry{WeightedRouletteSelector}
            \addplot +[mark=none] table [x=gen, y=tournament, col sep=comma] {./figures/selectors.csv};
            \addlegendentry{TournamentSelector}
        \end{axis}
    \end{tikzpicture}
    \caption{Μέσος όρος καλύτερου fitness value ανά γενιά}
    \label{fig_selectorFitness}
\end{figure}

\begin{figure}[!t]
    \centering
    \begin{tikzpicture}
    	\begin{axis}[
            xlabel={x},
            ylabel={y},
            width = 0.8\columnwidth,
            %height = 0.8\columnwidth,
            colorbar,
            legend cell align = left,
            legend pos = north west,
            %colormap = {whiteblack}{gray(0cm) = (1); gray(1cm) = (0)},
            colorbar style = {title=Αριθμός Σφηκών, /tikz/.cd},
            font = \footnotesize]

            \addplot +[scatter, only marks, point meta=explicit] table [meta=wasps, col sep=comma] {./figures/waspNests.csv};
            \addlegendentry{Σφηκοφωλιά}
        \end{axis}
    \end{tikzpicture}
    \caption{Χάρτης σφηκοφωλιών}
    \label{fig_waspNestsMap}
\end{figure} 